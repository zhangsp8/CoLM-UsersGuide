% !TEX encoding = UTF-8 Unicode
\documentclass[a4paper,12pt,twoside]{article}
\usepackage[left=2.5cm,right=2.5cm,top=3cm,bottom=3cm]{geometry}
\usepackage[dvipsnames]{xcolor}
\usepackage[UTF8,heading=true,scheme=chinese]{ctex}
\usepackage{fancyhdr}
%\usepackage[nottoc]{tocbibind} % used to add bibliography to toc with correct bookmark and page number

%\usepackage[T1]{fontenc}
\newcommand{\arial}[1]{{\fontencoding{T1}\usefont{T1}{phv}{m}{n}{#1}}}
\newcommand{\courier}[1]{{\fontencoding{T1}\usefont{T1}{pcr}{m}{n}{#1}}}
\newcommand{\Times}[1]{{\fontencoding{T1}\usefont{T1}{ptm}{m}{n}{#1}}}
\newcommand{\Rom}[1]{{\fontencoding{T1}\usefont{T1}{cmr}{m}{n}{#1}}}
% \usepackage{wasysym}

\input{CoLM.preamble}

\setlength{\parskip}{0em}
\setlength{\parindent}{2em}
\setcounter{secnumdepth}{3}
\usepackage{enumerate}
\usepackage{wasysym}
\usepackage{textcomp}
\usepackage{rotating}
\usepackage{amssymb}
\usepackage{afterpage} % avoid blank page before landscape environment
\usepackage{pdflscape}
\usepackage{xr}
\usepackage{color}
\usepackage{textcomp}
\usepackage{threeparttable}
\usepackage{array}
\usepackage{longtable} % for 'longtable' environment
\usepackage{threeparttablex} % for 'ThreePartTable' environment
\usepackage{makecell}
\usepackage{listings}
\definecolor{bg}{RGB}{240,240,240}

\lstset{
  basicstyle=\small\ttfamily,
%  columns=flexible,
  breaklines=true,
  backgroundcolor=\color{bg},
  frame = single,
  frameround=tttt,
  rulecolor=\color{bg}
}
\raggedbottom

\usepackage{siunitx}

\begin{document}

\title{\huge {\bf 通用陆面模式2024版用户手册}\\
\vspace{6mm}
\fontsize {22}{24}
\bf{ The Common Land Model Users' Guide}
\fontsize {20}{23}
 \vskip 2in
}

\author{
 \large{ 中山大学 }\\[2ex]
 {\bf 大气科学学院}
 \vskip 2in
 \upshape
 \large
 \vskip 0.5in
 2024年8月
}

\normallinespacing
\maketitle

\preface

\clearpage 
\pagestyle{fancy}
\tableofcontents
% \listoftables
% \listoffigures
\clearpage
\pagenumbering{arabic}

\section{引言}

陆面过程模式是地球系统模式的核心组成部分,包含了对复杂的物理、化学、生物以及人类活动过程的描述,具有多时空尺度、系统集成性强等特点,因此,模式代码需要实现多种功能。

数据处理方面,陆面模式代码可将多源的、具有不同分辨率的数据融合到一起。模式中的代表性数据包括米级分辨率的数字高程数据、公里级分辨率的地表特征数据(如土地利用、叶面积指数)和十公里级分辨率的大气驱动数据等。在模式中,通过升尺度或降尺度方法,将这些数据映射到模式网格单元进行使用。

代码结构方面,模式能够兼容多种网格和次网格划分方法,对陆地表面的异质性进行准确描述。除常用的经纬度网格外,通用陆面模式还提供了另外两种网格,一种是可根据陆表异质性进行任意加密的非结构网格,另一种是适用于山坡尺度陆面过程模拟的流域单元网格。在网格单元内部,通用陆面模式可根据地表覆盖类型、植被功能型或者植物群落对被自然植被覆盖的地表进行进一步的次网格划分。对上述多种网格和次网格划分方法,模式中的数据结构、升降尺度方法及物理、化学和生物过程方案基本都是共享的。

可移植性方面,模式可通过简单的配置,在个人计算机、小型服务器和超级计算机等平台上编译和运行。通用陆面模式的运行仅需Fortran环境和NetCDF软件,并可通过MPI在并行平台上对模拟进行加速。

应用方面,模式需容易操作、有较好的可配置性,同时代码易读易改。一般用户可快速编译和运行一个模式实例,也可通过配置文件满足多样性的需求。高级用户可在通用陆面模式平台上,对模式代码进行增加和修改,满足更广泛和深入的研究需求。

本书介绍了通用陆面模式CoLM的使用方法、代码规范和软件设计理念,可作为一般用户使用模式的入门指南,也可作为高级用户深度使用通用陆面模式的工具书。


\section{快速运行一个模式实例}\label{chapter01}

\subsection{代码基本结构}

CoLM包含三个主要的程序:制作地表数据、制作初始场数据和主程序。三个程序是相互独立的,但需按顺序依次执行。

制作地表数据是指构建模式网格和次网格单元,以及由高分辨的原始数据聚合得到模式单元上的地表属性。CoLM可使用经纬度单元、非结构单元、流域单元和单点四种模式网格单元,网格单元进一步再细分为植被、城市、湿地、冰川和水体五大类次网格单元,其中,植被次网格单元可选择地表覆盖类型、植被功能型和植物群落三种网格植被结构中的一种进行表征。次网格是CoLM计算模拟的基本结构单元,在运行主程序前,需首先由高分辨率的原始数据进行升尺度获取模式单元上的地表属性。原始地表数据通常为1公里及以下的高分辨率数据,包括地表覆盖类型、植被结构及属性、土壤属性、高程数据、水文学数据和城市数据等。

制作初始场数据是指构建模式的初始状态。初始状态分为冷启动和热启动两种,其中,冷启动指的是设定一个物理上合理的土壤水、土壤温度和积雪状态等数值,热启动指的是由外部数据读入已经运行过一段时间后的模式状态。冷启动设定简单,不需要额外的数据,但通常主程序需运行比热启动更长的时间,才能达到较为平衡的状态。

主程序对陆面主要的物理、化学、生物和人类活动等过程进行时间积分预报。主程序分离线运行和与大气模式耦合运行两种情况:离线运行时,需准备好大气驱动数据作为输入;耦合运行时,模式以一定的频率从大气模式在线获取驱动数据。此外,根据模式功能,可能还需要气溶胶、氮沉降和臭氧等数据作为输入,在运行时在线读入。

CoLM程序代码的子目录见表~\ref{subdirectories}。
\begin{table}[!htbp]
\caption{CoLM目录} \label{subdirectories}
\centering \renewcommand{\arraystretch}{1.5}
\begin{tabular}{cp{0.8\textwidth}}
\toprule
\textbf{目录名称} & \textbf{说明} \\
\midrule
\texttt{include} & 包含编译选项\texttt{Makeoptions}文件和预处理宏定义文件\texttt{define.h} \\
\texttt{share} & 包含模式中的部分常量、模式数据结构模块、共享的函数和输入输出模块等 \\
\texttt{mksrfdata} & 制作地表数据;生成可执行文件\texttt{mksrfdata.x};可独立运行 \\
\texttt{mkinidata} & 制作初始场数据;生成可执行文件\texttt{mkinidata.x};需在制作完地表数据后运行\\
\texttt{main} & 模式主程序;生成可执行文件\texttt{colm.x};需在制作完地表数据和初始场数据后运行\\
\texttt{CaMa} & \texttt{CaMa-Flood}径流模型代码和数据 \\
\texttt{run} & 存放可执行文件和namelist文件 \\
\texttt{postprocess} & 后处理程序,主要用于将分块输出的变量文件合并成一个整体 \\
\bottomrule
\end{tabular}
\end{table}

\subsection{编译和运行}\label{comprun}

CoLM的编译和运行分为软件环境的配置、数据的准备和模式运行三个主要步骤。假设代码放置的根目录为 \texttt{\$CoLMRoot}.

\textbf{第1步},软件环境的配置。

CoLM在Linux操作系统下运行,其软件需求为:
\begin{quote}
\begin{itemize}
\setlength{\itemsep}{0pt}
\setlength{\parsep}{0pt}
\setlength{\parskip}{0pt}
    \item Fortran编译器;
    \item MPI(Message Passing Interface)软件包;
    \item NetCDF(Network Common Data Form)软件包;
    \item LAPACK(Linear Algebra PACKage)软件包;
    \item BLAS(Basic Linear Algebra Subprograms)软件包。
\end{itemize}
\end{quote}

以下文件内容给出了一个典型环境下的软件配置,文件位于\texttt{\$CoLMRoot/include/\allowbreak Makeopitons}:
\begin{lstlisting}[language=bash, basicstyle=\linespread{1.2}\small\ttfamily, commentstyle=\color{olive}, numbers=left, numberstyle=\tiny, xleftmargin=1.5em,xrightmargin=0em, aboveskip=1em]
# An example for file '$CoLMRoot/include/Makeoptions'.

# 设置编译器
FF = mpif90

# 设置NetCDF软件包的路径
NETCDF_LIB = /usr/lib/x86_64-linux-gnu
NETCDF_INC = /usr/include

MOD_CMD = -J

FOPTS = -fdefault-real-8 -ffree-form -C -g -u -xcheck=stkovf \
        -ffpe-trap=invalid,zero,overflow -fbounds-check      \
        -mcmodel=medium -fbacktrace -fdump-core -cpp         \
        -ffree-line-length-0 -fallow-argument-mismatch

INCLUDE_DIR = -I../include -I../share -I../mksrfdata  \ 
              -I../mkinidata -I../main -I$(NETCDF_INC)
LDFLAGS = -L$(NETCDF_LIB) -lnetcdff -lnetcdf -llapack -lblas

\end{lstlisting}

以上文件对编译器、NetCDF软件包的路径和编译选项等进行了配置。其中,编译器mpif90为系统默认的集成MPI环境的Fortran编译器,可在Linux系统中使用命令 \texttt{which mpif90} 查看其完整路径。文件中使用了NETCDF\_LIB和NETCDF\_INC两个变量对NetCDF软件包的路径进行了设置。LAPACK和BLAS是常用的软件包,这里并未对其路径进行设置,编译器会在系统默认的软件包路径中进行查找。通常来讲,用户只需确认编译器和NetCDF软件包的设置正确,即可对CoLM进行编译。

\textbf{第2步},准备数据。

CoLM的运行需要两个必要数据:地表属性数据和大气驱动数据。

模式单元上的地表属性数据可由高分辨率的原始数据升尺度得到,也可直接使用已经制作好的数据。若使用原始数据升尺度,需准备好完整的原始数据,这些数据大多为全球1公里及以下分辨率的数据,体量较大。对一些常用的网格单元,如0.5度的经纬度网格,CoLM预先制作好了地表属性数据,可直接下载使用。

当进行离线模拟时,还需准备大气驱动数据。CoLM支持十几种常见的离线驱动数据,这些数据可从其数据网站上下载,然后使用 \texttt{\$CoLMRoot/preprocess/Forcings} 目录下的对应预处理程序转换为CoLM可直接使用的数据。

\textbf{第3步},运行模式。

CoLM通过以下两个文件对一个模式实例进行设置,
\begin{quote}
\begin{enumerate}[1)]
    \item \textbf{\$CoLMRoot/include/define.h}:模式主要模块和功能的选择或开关
    \item \textbf{\$CoLMRoot/run/\$NamelistFile}:模式运行的具体设置
\end{enumerate}
\end{quote}

一个最简单的define.h文件的示例如下,
\begin{lstlisting}[language=fortran, basicstyle=\linespread{1.2}\small\ttfamily, commentstyle=\color{olive}, numbers=left, numberstyle=\tiny, xleftmargin=1.5em,xrightmargin=0em, aboveskip=1em]
! 使用“经纬度网格”单元
#define GRIDBASED
! 使用“地表覆盖类型”表征次网格单元
#define LULC_IGBP
! 使用MPI进行并行加速
#define USEMPI
! 使用van Genuchten-Mualem土壤水模型
#define  vanGenuchten_Mualem_SOIL_MODEL
\end{lstlisting}

以上文件设置模式使用经纬度网格、“地表覆盖类型”次网格、使用MPI软件进行并行加速和van Genuchten-Mualem土壤水模型。

一个最简单的namelist文件示例如下,
\begin{lstlisting}[language=fortran, basicstyle=\linespread{1.2}\small\ttfamily, commentstyle=\color{olive}, numbers=left, numberstyle=\tiny, xleftmargin=1.5em,xrightmargin=0em, aboveskip=1em]
&nl_colm

   ! 设定实例名称
   DEF_CASE_NAME = 'GreaterBay_Grid_10km_IGBP_VG'

   ! 设置模拟的区域
   DEF_domain%edges = 20.0
   DEF_domain%edgen = 25.0
   DEF_domain%edgew = 109.0
   DEF_domain%edgee = 118.0

   ! 设置模拟的起止时间,预热的时间及重复次数
   DEF_simulation_time%greenwich     = .TRUE.
   DEF_simulation_time%start_year    = 2010
   DEF_simulation_time%start_month   = 1
   DEF_simulation_time%start_day     = 1
   DEF_simulation_time%start_sec     = 0
   DEF_simulation_time%end_year      = 2010
   DEF_simulation_time%end_month     = 1
   DEF_simulation_time%end_day       = 31
   DEF_simulation_time%end_sec       = 86400
   DEF_simulation_time%spinup_year   = 2010
   DEF_simulation_time%spinup_month  = 1
   DEF_simulation_time%spinup_day    = 10
   DEF_simulation_time%spinup_sec    = 86400
   DEF_simulation_time%spinup_repeat = 1

   ! 设置模拟的时间步长
   DEF_simulation_time%timestep     = 1800.

   ! 设置数据路径
   !  原始高分辨率地表属性数据路径
   DEF_dir_rawdata = '/path/to/rawdata/'   
   !  运行时需要用到的数据路径
   DEF_dir_runtime = '/path/to/runtime/data/'   
   !  输出数据存放的路径
   DEF_dir_output  = '/path/to/output/directory'  

   ! 设置经纬度网格单元分辨率
   DEF_GRIDBASED_lon_res = 0.5
   DEF_GRIDBASED_lat_res = 0.5

   ! 设置驱动数据的namelist文件路径
   DEF_forcing_namelist = '/path/to/forcing.nml'

   ! 历史数据输出设置
   DEF_HIST_lon_res = 0.5     ! 输出数据经向分辨率
   DEF_HIST_lat_res = 0.5     ! 输出数据纬向分辨率
   DEF_HIST_FREQ = 'DAILY'    ! 输出数据的日平均值
   DEF_HIST_groupby = 'MONTH' ! 每月的数据放置在一个文件中
   DEF_hist_vars_out_default = .true. ! 默认输出所有变量

   ! 重启动数据输出设置
   DEF_WRST_FREQ = 'MONTHLY' ! 每月保存一次模式状态

/
\end{lstlisting}

以上文件设置:模拟区域为覆盖大湾区的矩形区域;起止时间为2010年1月1日至2010年1月31日,其中前10天为预热时间;模拟的时间步长为半小时;模拟分辨率为0.5度;历史数据的输出分辨率为0.5度,默认输出所有变量的日平均值,且将每月的数据放置于一个文件内;每月保存一次模式状态用于重启动;同时,对输入输出数据进行了设置。

完成软件环境的配置、数据的准备和模式实例的配置后,通过执行以下命令进行CoLM的编译,
\begin{quote}
\begin{lstlisting}
cd $CoLMRoot && make
\end{lstlisting}
\end{quote}
编译完成后,会在\texttt{\$CoLMRoot/run}目录下生成\texttt{mksrfdata.x}, \texttt{mkinidata.x}, \texttt{colm.x} 三个可执行文件,依次执行这三个文件进行地表数据的制作、初始场数据的制作和主程序的运行。例如,使用mpirun进行运行时,可依次执行
\begin{quote}\label{runcolm}
\begin{lstlisting}
mpirun -np $np $CoLMRoot/run/mksrfdata.x $NamelistFile
mpirun -np $np $CoLMRoot/run/mkinidata.x $NamelistFile
mpirun -np $np $CoLMRoot/run/colm.x $NamelistFile
\end{lstlisting}
\end{quote}
其中,\verb|$np| 为调用的进程数,\verb|$NamelistFile| 为配置模式实例的 namelist 文件。

模式的输出数据包含分别为地表属性数据、模式状态数据和历史输出数据三部分,分别放置于输出目录的子文件夹 \texttt{landdata},\texttt{restart} 和 \texttt{history}中。

\section{编译时配置:宏定义文件define.h}\label{define.hux6587ux4ef6}

模式使用了条件编译来实现各种模块或功能的打开和关闭,用户可打开\texttt{include/\allowbreak define.h}文件进行修改,可使用的模块和功能选项见表~\ref{table_define}。
\begin{longtable}{ll}
\caption{define.h中的模块/功能选项} \label{table_define} \\
\toprule
\textbf{模块/功能选项} & \textbf{说明} \\
\midrule
\endfirsthead
\multicolumn{2}{r}
{{\bfseries \tablename\ \thetable{} -- \kaishu 续表}} \\
\toprule
\textbf{模块/功能选项} & \textbf{说明} \\
\midrule
\endhead
\bottomrule
\endfoot
\bottomrule
\endlastfoot
\colorbox{gray}{\textcolor{white}{\bf{四选一}}} & \bf{网格结构} \\
GRIDBASED & 经纬度网格 \\
CATCHMENT & 流域单元网格 \\
UNSTRUCTURED & 非结构网格(三角形/六边形)\\
SinglePoint & 单点模式 \\
\hline
\colorbox{gray}{\textcolor{white}{\bf{四选一}}} & \bf{植被次网格结构} \\ 
LULC\_USGS & 基于地表覆盖类型(USGS分类) \\
LULC\_IGBP & 基于地表覆盖类型(IGBP分类) \\
LULC\_PFT & 基于植被功能类型(PFT) \\
LULC\_PC & 基于植被群落(PC) \\
\hline
\colorbox{gray}{\textcolor{white}{\bf{开关}}} & \bf{城市模块} \\
URBAN\_MODEL & 是否打开城市冠层模式 \\
\hline
\colorbox{gray}{\textcolor{white}{\bf{二选一}}} & \bf{土壤水分特征模型} \\
Campbell\_SOIL\_MODEL &  Campbell模型 \\
vanGenuchten\_Mualem\_SOIL\_MODEL & van Genuchten \& Mualem模型 \\
\hline
\colorbox{gray}{\textcolor{white}{\bf{开关}}} & \bf{侧向流模块} \\
LATERAL\_FLOW &  是否模拟侧向流过程 \\
\hline
\colorbox{gray}{\textcolor{white}{\bf{开关}}} & \bf{CaMa-Flood径流模型} \\
CaMa\_Flood &  是否耦合CaMa-Flood径流模型 \\
\hline
\colorbox{gray}{\textcolor{white}{\bf{开关}}} & \bf{生物地球化学模块} \\
BGC &  是否模拟生物地球化学过程 \\
\hline
\colorbox{gray}{\textcolor{white}{\bf{开关}}} & \bf{作物模块} \\
CROP &  是否模拟作物过程 \\
\hline
\colorbox{gray}{\textcolor{white}{\bf{开关}}} & \bf{土地利用土地覆盖变化模块} \\
LULCC &  是否模拟土地利用和土地覆盖变化过程 \\
\hline
\colorbox{gray}{\textcolor{white}{\bf{开关}}} & \bf{数据同化模块} \\
DataAssimilation &  是否同化观测数据 \\
\hline
\colorbox{gray}{\textcolor{white}{\bf{开关}}} & \bf{MPI并行} \\
USEMPI &  是否使用MPI程序进行并行 \\
\hline
\colorbox{gray}{\textcolor{white}{\bf{开关}}} & \bf{检查变量数值范围} \\
RangeCheck &  是否检查变量数值范围 \\
\hline
\colorbox{gray}{\textcolor{white}{\bf{开关}}} & \bf{代码调试} \\
CoLMDEBUG &  是否显示模式调试信息 \\
\hline
\colorbox{gray}{\textcolor{white}{\bf{开关}}} & \bf{陆面属性数据诊断输出} \\
SrfdataDiag &  \makecell[l]{是否对聚合得到的陆面属性数据\\进行诊断输出} \\

\end{longtable}

\section{运行时配置:Namelist文件}\label{nml}

\subsection{实例的基本信息}
运行实例的基本信息包括实例名称、模拟区域的范围、模拟的时间范围,这些信息在namelist文件中设置,变量见表~\ref{table_nl_basic}。

\begin{table}[!htbp]
\caption{Namelist变量:模拟实例的基本信息设置} \label{table_nl_basic}
\centering \renewcommand{\arraystretch}{1.2}
\begin{tabular}{lcp{0.3\textwidth}}
\toprule
\textbf{变量名} & \textbf{数据类型} & \textbf{说明} \\
DEF\_CASE\_NAME & 字符串 & 定义了实例的名称 \\
 \\\midrule
\textbf{模拟区域} && \\
DEF\_domain & 自定义类型 & 定义了模拟区域的边界 \\
DEF\_domain\%edges & 浮点型 & 定义了模拟区域的南边界 \\
DEF\_domain\%edgen & 浮点型 & 定义了模拟区域的北边界 \\
DEF\_domain\%edgew & 浮点型 & 定义了模拟区域的西边界 \\
DEF\_domain\%edgee & 浮点型 & 定义了模拟区域的东边界 \\\midrule
\textbf{模拟时间} && \\
DEF\_simulation\_time & 自定义类型 & 模拟的时间 \\
DEF\_simulation\_time\%greenwich & 逻辑型 & 是否使用格林威治时间 \\
DEF\_simulation\_time\%start\_year & 整型 & 起始年份 \\
DEF\_simulation\_time\%start\_month & 整型 & 起始月份 \\
DEF\_simulation\_time\%start\_day & 整型 & 起始月份中的第几天 \\
DEF\_simulation\_time\%start\_sec & 整型 & 起始天中的第几秒 \\
DEF\_simulation\_time\%end\_year & 整型 & 终止年份 \\
DEF\_simulation\_time\%end\_month & 整型 & 终止月份 \\
DEF\_simulation\_time\%end\_day & 整型 & 终止月份中的第几天 \\
DEF\_simulation\_time\%end\_sec & 整型 & 终止天中的第几秒 \\
DEF\_simulation\_time\%spinup\_year & 整型 & 模式预热结束年份 \\
DEF\_simulation\_time\%spinup\_month & 整型 & 模式预热结束月份 \\
DEF\_simulation\_time\%spinup\_day & 整型 & 模式预热结束月份中的第几天 \\
DEF\_simulation\_time\%spinup\_sec & 整型 & 模式预热结束天中的第几秒 \\
DEF\_simulation\_time\%spinup\_repeat & 整型 & 模式预热重复的次数 \\
DEF\_simulation\_time\%timestep & 浮点型 & 模拟的时间步长 \\
\bottomrule
\end{tabular} 
\end{table}

\subsection{数据目录}
\begin{table}[!htbp]
\caption{Namelist变量:数据目录及文件} \label{table_nl_dir_file}
\centering \renewcommand{\arraystretch}{1.2}
\begin{tabular}{lcp{0.5\textwidth}}
\toprule
\textbf{变量名} & \textbf{数据类型} & \textbf{说明} \\
DEF\_dir\_rawdata & 字符串 & 高分辨率地表属性数据路径 \\
DEF\_dir\_runtime & 字符串 & 模式运行或初始化时的输入数据路径 \\
DEF\_dir\_output & 字符串 & 模式输出数据路径:在此目录下自动建立一个与实例名称同名的目录,其下再建立三个子目录:landdata, restart, history,分别用于存放本实例制作好的地表数据,重启动数据文件,变量的历史文件 \\
DEF\_DA\_obsdir & 字符串 & 进行数据同化时,观测数据存放的目录 \\
\bottomrule
\end{tabular} 
\end{table}

\subsection{数据分块及MPI并行}

为了处理高分辨率的数据和进行并行计算,CoLM将全球分成二维的块,模式中所有数据都会被分配到块上。分块信息由DEF\_nx\_blocks和DEF\_ny\_blocks设定,在制作地表数据、生成初始场以及运行模式主程序的过程中都是不变的。运行模式主程序时,大气驱动场会以分块的方式被读入内存。模式变量输出时,可选择以分块的方式输出,以提高写入硬盘的速度,也可选择在程序内部进行合并后输出,以避免再次进行后处理。若在区域上运行模式,分块也是在全球进行的,此时,只有与区域有交集的块是有效的。

CoLM可以采用串行方式运行,方便对代码进行调试;也可以并行方式运行,以实现大规模、高分辨率的计算。

模式以MPI并行方式执行时,将所有进程分为三类。第一类为主进程(Master),只有1个进程为主进程,它负责读入全局的信息(如namelist文件),和输出全局信息(如调试信息);第二类为输入输出进程(IO),这些进程执行数据的读入和写出,每个数据块都被关联到唯一的一个IO进程,为了达到负载均衡,模式采用轮询的方式将数据块关联到IO进程上;第三类为工作进程(Worker),这些进程负责主要的模式计算。

为了减少IO进程与Worker进程间的数据交换量,模式将每个IO进程与多个Worker进程进行了绑定,IO进程负责的数据块上的计算任务会分发给与之绑定的Worker进程。一个IO进程和与之绑定的多个Worker进程组成一个进程组,进程组的大小可在模式运行时由DEF\_PIO\_groupsize进行设定。

数据分块及MPI并行相关namelist文件中变量见表~\ref{table_nl_blocks_mpi}。

\begin{table}[!htbp]
\caption{Namelist变量:数据分块及MPI并行} \label{table_nl_blocks_mpi}
\centering \renewcommand{\arraystretch}{1.5}
\begin{tabular}{lcl}
\toprule
\textbf{变量名} & \textbf{数据类型} & \textbf{说明} \\
DEF\_nx\_blocks & 整型 & 经度方向全球分块的数目(要求可被360整除)\\
DEF\_ny\_blocks & 整型 & 纬度方向全球分块的数目(要求可被180整除)\\
DEF\_PIO\_groupsize & 整型 & 一个进程组中进程的数量 \\
\bottomrule
\end{tabular} 
\end{table}

\subsection{模式空间单元结构}

CoLM可以在三种陆表单元结构下运行:1)经纬度网格结构;2)非结构网格;3)流域单元网格。此外,模式还可以在单点上运行。

考虑到地表下垫面覆盖的异质性,CoLM对模式网格单元进一步划分次网格,次网格可根据地表覆盖类型(LCT)、植被功能类型(PFT)和植被群落(PC)进行划分。采用地表覆盖类型进行划分时,可选择基于USGS分类或者IGBP分类。

\texttt{define.h}中的相关选项见表~\ref{table_define}。

在namelist文件中,对定义模式空间单元结构使用的数据进行设置。在使用经纬度网格时,可选择从文件读入网格的定义数据(DEF\_file\_mesh)或者直接指定网格单元的分辨率(设置DEF\_GRIDBASED\_lon\_res和DEF\_GRIDBASED\_lat\_res)。在使用流域单元网格和非结构网格时,需要从文件读入数据。此外,使用可选的DEF\_file\_mesh\_\allowbreak filter文件可从由DEF\_domain定义的模拟区域中屏蔽掉部分子区域或者网格。

次网格选用主要通过\texttt{define.h}进行设置,同时可以在namelist文件进行部分选项细化,包括DEF\_FAST\_PC和DEF\_SOLO\_PFT。当DEF\_FAST\_PC设置为 true 时,每个网格中所有PFT看成一个植物群落进行模拟,此时会降低模型计算量,提高模拟速度(FAST);为 false 时,网格中每种地表覆盖类型中的PFT各自看成一个植物群落进行模拟。DEF\_SOLO\_PFT默认为 false,当设置为 true 时,单独 (SOLO) 模拟网格中每种地表覆盖所包含的 PFT 类型,此选项模拟相比而言更加精细化,但同时会增加计算量。PFT 次网格方案默认设置为false,即把网格内所有同种 PFT 进行聚合后模拟 。

Namelist文件中的相关选项见表~\ref{table_nl_structure}。

\begin{table}[!htbp]
\caption{Namelist变量:模式空间结构相关变量} \label{table_nl_structure}
\centering \renewcommand{\arraystretch}{1.2}
\begin{tabular}{lcp{0.35\textwidth}}
\toprule
\textbf{1.模式网格单元} && \\
\textbf{变量名} & \textbf{数据类型} & \textbf{说明} \\DEF\_file\_mesh & 字符串 & 划分\textbf{经纬度网格单元}或\textbf{非结构网格单元}的数据文件 \\
DEF\_CatchmentMesh\_data & 字符串 & 划分\textbf{流域网格单元}的数据文件或路径 \\
DEF\_GRIDBASED\_lon\_res & 浮点型 & 定义了\textbf{经纬度网格单元}的经向分辨率,当DEF\_file\_mesh指向的文件不存在时有效 \\
DEF\_GRIDBASED\_lat\_res & 浮点型 & 定义了\textbf{经纬度网格单元}的纬向分辨率,当DEF\_file\_mesh指向的文件不存在时有效 \\
DEF\_file\_mesh\_filter & 字符串 & 使用此文件,可从模拟区域中屏蔽掉部分子区域或格点 \\
\midrule
\textbf{2. 次网格单元} & & \\ 
\textbf{变量名} & \textbf{数据类型} & \textbf{说明} \\
DEF\_USE\_USGS & 逻辑型 & 通过宏定义进行预设,用户无需设置\\
DEF\_USE\_IGBP & 逻辑型 & 通过宏定义进行预设,用户无需设置\\
DEF\_USE\_LCT & 逻辑型 & 通过宏定义进行预设,用户无需设置\\
DEF\_USE\_PFT & 逻辑型 & 通过宏定义进行预设,用户无需设置\\
DEF\_USE\_PC & 逻辑型 & 通过宏定义进行预设,用户无需设置\\
DEF\_SOLO\_PFT & 逻辑型 & 默认为false。当设置为true时,单独(SOLO)模拟网格中每种地表覆盖所包含的PFT类型;为false时,把网格内所有同种 PFT进行聚合后进行模拟\\
DEF\_FAST\_PC & 逻辑型 & 设置为true时,每个网格中所包含的PFT看成一个植物群落进行模拟,此选项会减少计算量,提高模拟速度(FAST);为false时,网格中每种地表覆盖类型中的PFT各自看成一个植物群落进行模拟\\
DEF\_SUBGRID\_SCHEME & 字符串 & 通过宏定义进行预设,用户无需设置 \\
\makecell[l]{DEF\_USE\_DOMINANT\_\\PATCHTYPE} & 逻辑型 & \\
\bottomrule
\end{tabular} 
\end{table}

\subsection{制作地表数据}

在模式运行前,需制作地表数据,即从高分辨率的数据聚合得到模式网格单元内的变量或者参数。对于制作的的地表数据需要指定年份,可通过namelist DEF\_LC\_YEAR进行设定,默认为2005年。当模拟区域较大时,从源数据制作地表数据通常需要使用较大的存储和内存空间,也需较长时间。模式中提供了从制作好的更大区域的数据中提取模拟区域的数据的方法,namelist文件中的相关选项设置见表~\ref{table_nl_mksrfdata}。

通过设置DEF\_LANDONLY为TRUE,可屏蔽掉模拟区域中的海洋区域。

对土壤参数的聚合,可选择使用简单算法(面积加权、中位数等)或者FIT升尺度算法,使用DEF\_USE\_SOILPAR\_UPS\_FIT进行设定。

土壤颜色的数值可通过基于地表覆盖类型的查找表给定或者从全球格点数据中读取,使用DEF\_SOIL\_REFL\_SCHEME进行选择。

此外,namelist文件中也有一些相关选项对制作地表数据的过程进行调优,见表~\ref{table_nl_mksrfdata}。
\newline
\newline
\noindent \colorbox{gray}{\textcolor{white}{\bf{注}}} 当打开LULCC模块时,目前需要依次手动设定地表数据年份,制作所需模拟时间跨度的地表数据。

\begin{table}[!htbp]
\caption{Namelist变量:制作地表数据相关变量} \label{table_nl_mksrfdata}
\centering \renewcommand{\arraystretch}{1.2}
\begin{tabular}{lcp{0.4\textwidth}}
\toprule
\textbf{变量名} & \textbf{数据类型} & \textbf{说明} \\\midrule
DEF\_LC\_YEAR & 整型 & 制作地表数据所采用的地表覆盖数据年份,默认为2005\\
USE\_srfdata\_from\_larger\_region & 逻辑型 & 当已有一个制作好的更大模拟区域的地表数据时,可从中提取出需要的小区域的数据 \\
DEF\_dir\_existing\_srfdata & 字符串 & 已有制作好的更大模拟区域地表数据的存放路径 \\
DEF\_LANDONLY & 逻辑型 & 值为TRUE,屏蔽掉海洋区域 \\
DEF\_USE\_SOILPAR\_UPS\_FIT & 逻辑型 & 是否使用FIT算法聚合土壤参数数据 \\
DEF\_SOIL\_REFL\_SCHEME & 整型 & 数值为1时,根据土地覆盖类型,使用查找表给定土壤颜色的类型;数值为2时,从全球格点数据中读取土壤颜色数值\\ 
USE\_zip\_for\_aggregation & 逻辑型 & 在并行模式下制作地表数据时,若像素点网格(约为模式使用的最高分辨率的数据网格,如90米)比高分辨率的数据网格(如土壤属性数据的1公里网格)更小,可将此变量设置为true,提高进程间数据传输的速度,默认为true,一般不需更改 \\
\bottomrule
\end{tabular} 
\end{table}



\subsection{单点模拟}

对单点进行模拟时,可以用站点的地表数据替代由全球格点数据中提取的数据。站点地表数据存放于文件\texttt{SITE\_fsrfdata}中,在\texttt{namelist}文件中,使用变量\texttt{USE\_SITE\_\allowbreak\$data}来指定数据来源,详见表~\ref{table_nl_singlepoint}。

使用站点的气象观测资料驱动模式时,可将驱动数据一次读入内存来提高运行速度,在\texttt{namelist}文件中,使用变量\texttt{USE\_SITE\_ForcingReadAhead}来打开这一功能。

对变量历史进行输出时,也可将输出内容保存于内存中,待一个文件的内容存满后一次性输出,在\texttt{namelist}文件中,使用变量\texttt{USE\_SITE\_HistWriteBack}来打开这一功能。

\begin{table}[!htbp]
\caption{Namelist变量:单点模拟}
\label{table_nl_singlepoint}
\centering \renewcommand{\arraystretch}{1.5}
\begin{tabular}{lcp{0.45\textwidth}}
\toprule
\textbf{1.数据文件} & & \\
\textbf{变量名} & \textbf{数据类型} & \textbf{说明} \\
SITE\_fsrfdata & 字符串 & 站点地表数据存放的文件 \\\midrule
\multicolumn{3}{l}{\textbf{2. 数据来源}(\textbf{TRUE}代表站点观测数据,\textbf{FALSE}代表全球格点数据)} \\
\textbf{变量名} & \textbf{数据类型} & \textbf{说明} \\
USE\_SITE\_pctpfts & 逻辑型 & 植被百分比数据来源 \\
& & (使用站点数据或全球格点数据,下同) \\
USE\_SITE\_pctcrop & 逻辑型 & 作物百分比数据来源 \\
USE\_SITE\_htop & 逻辑型 & 树高数据来源 \\
USE\_SITE\_LAI & 逻辑型 & 叶面积指数数据来源 \\
USE\_SITE\_lakedepth & 逻辑型 & 湖泊深度数据来源 \\
USE\_SITE\_soilreflectance & 逻辑型 & 土壤反射数据来源 \\
USE\_SITE\_soilparameters & 逻辑型 & 土壤水力和热力参数数据来源 \\
USE\_SITE\_dbedrock & 逻辑型 & 基岩深度数据来源 \\
USE\_SITE\_topography & 逻辑型 & 高程数据来源 \\
USE\_SITE\_urban\_paras & 逻辑型 & 城市参数数据来源 \\
& &(包括建筑形态、植被及水体参数) \\
USE\_SITE\_thermal\_paras & 逻辑型 & 热力参数数据来源 \\
USE\_SITE\_urban\_LAI & 逻辑型 & 城市叶面积指数数据来源 \\
\midrule
\textbf{3.运行时的优化} & & \\
\textbf{变量名} & \textbf{数据类型} & \textbf{说明} \\
USE\_SITE\_ForcingReadAhead & 逻辑型 & 是否预先将所有大气驱动数据读入内存 \\
USE\_SITE\_HistWriteBack & 逻辑型 & 是否将输出内容暂存于内存中,待一个文件存满后一次性输出 \\
\bottomrule
\end{tabular} 
\end{table}


\subsection{叶面积指数}

CoLM可使用固定年、多年平均或者逐年变化的叶面积指数数据,在namelist文件中使用DEF\_LAI\_CHANGE\_YEARLY进行选择。

CoLM可使用月值或者每8天(卫星观测的周期)的叶面积指数数据,在namelist文件中使用DEF\_LAI\_MONTHLY进行选择。

具体设置见表~\ref{table_nl_lai}。

\begin{table}[!htbp]
\caption{Namelist变量:叶面积指数}
\label{table_nl_lai}
\centering \renewcommand{\arraystretch}{1.5}
\begin{tabular}{lcp{0.45\textwidth}}
\toprule
\textbf{变量名} & \textbf{数据类型} & \textbf{说明} \\\midrule
DEF\_LAI\_MONTHLY & 逻辑型 & 使用叶面积指数月值数据(TRUE)或每8天的数据(FALSE)\\
DEF\_LAI\_CHANGE\_YEARLY  & 逻辑型 & 是否使用逐年变化的叶面积指数数据 \\
DEF\_USE\_LAIFEEDBACK & 逻辑型 & 使用叶碳含量计算叶面积指数(TRUE)或使用卫星遥感的叶面积产品(FALSE) \\
\bottomrule
\end{tabular} 
\end{table}
\clearpage

\subsection{状态变量初始化}
\begin{table}[!htbp]
\caption{Namelist变量:状态变量初始化}
\label{table_nl_ini}
\centering \renewcommand{\arraystretch}{1.5}
\begin{tabular}{lcp{0.5\textwidth}}
\toprule
\textbf{变量名} & \textbf{数据类型} & \textbf{说明} \\\midrule
DEF\_USE\_SoilInit & 逻辑型 & 是否使用数据对土壤水热状态进行初始化 \\
DEF\_file\_SoilInit & 字符串 & 土壤水热状态数据存放的文件 \\
DEF\_USE\_SnowInit & 逻辑型 & 是否使用数据对积雪的状态进行初始化 \\
DEF\_file\_SnowInit & 字符串 & 积雪状态数据存放的文件 \\
DEF\_USE\_CN\_Init & 逻辑型 & 是否用数据对植被土壤碳氮库进行初始化 \\
DEF\_file\_cn\_Init & 字符串 & 植被土壤碳氮库数据存放的文件 \\
\bottomrule
\end{tabular} 
\end{table}

\subsection{方案选项}

\begin{longtable}[!htbp]{lcp{0.45\textwidth}}
\caption{Namelist变量:运行时选项} \\
\toprule
\textbf{变量名} & \textbf{数据类型} & \textbf{说明} \\\midrule
\endfirsthead

\multicolumn{3}{r}
{{\bfseries \tablename\ \thetable{} -- \kaishu 续表}} \\
\toprule
\textbf{变量名} & \textbf{数据类型} & \textbf{说明} \\\midrule
\endhead

\bottomrule
\endfoot
\bottomrule
\endlastfoot
\textbf{1. 数据相关} && \\
DEF\_NDEP\_FREQUENCY & 整型 & 大气氮沉降数据读入方式:数值为1表示每年读入一次;数值为2表示每月读入一次 \\
DEF\_Aerosol\_Readin & 逻辑型 & 是否从文件中读入气溶胶沉降数据 \\
DEF\_Aerosol\_Clim & 逻辑型 & 读入气候态的气溶胶数据(TRUE)或者逐年变化的数据(FALSE)\\ 
\midrule
\textbf{2.参数化方案相关} && \\
DEF\_Runoff\_SCHEME & 整型 & 产流方案选项:0:SIMTOP; 1:VIC; 2:Xinanjiang; 3:Simple-VIC\\
DEF\_LULCC\_SCHEME & 整型 & 使用土地利用土地覆盖变化模块时状态变量的转换方案选项:1代表同类型赋值方案(STA);2代表质量和能量守恒转换方案(MEC) \\
DEF\_Interception\_scheme & 整型 & 冠层截留参数化方案。数值分别表示:1:CoLM;2:CLM4.5; 3:CLM5; 4:Noah-MP; 5:MATSIRO; 6:VIC; 7:JULES\\
\multicolumn{3}{l}{DEF\_THERMAL\_CONDUCTIVITY\_SCHEME} \\
 (接上行)& 整型 & 土壤热导率方案,数值分别代表:1: Farouki (1981); 2: Johansen(1975); 3: Cote and Konrad (2005); 4: Balland and Arp (2005); 5: Lu et al. (2007); 6: Tarnawski and Leong (2012); 7: De Vries (1963); 8: Yan Hengnian, He Hailong et al.(2019) \\
\multicolumn{3}{l}{DEF\_USE\_SUPERCOOL\_WATER} \\
 (接上行)& 逻辑型 & 是否使用过冷土壤水方案Niu \& Yang, 2006\\
DEF\_RSS\_SCHEME & 整型 & 土壤阻抗方案,数值分别代表:0: 不考虑土壤阻抗; 1: SL14, Swenson and Lawrence (2014); 2: SZ09, Sakaguchi and Zeng (2009); 3: TR13, Tang and Riley (2013); 4: LP92, Lee and Pielke (1992); 5: S92,  Sellers et al (1992) \\
DEF\_SPLIT\_SOILSNOW & 逻辑型 & 数值为TRUE时,分开模拟最上层的土壤和积雪,包括太阳吸收、感热/潜热、地表温度(包含相变)、地热通量和地面蒸发/凝结/升华/冻结等过程,该方案会直接影响季节性积雪和融雪过程 \\
\multicolumn{3}{l}{DEF\_USE\_VARIABLY\_SATURATED\_FLOW} \\ (接上行)& 逻辑型 & 是否使用变饱和土壤水运动算法\\
DEF\_USE\_OZONESTRESS & 逻辑型 & 是否开启臭氧胁迫方案 \\
DEF\_USE\_OZONEDATA & 逻辑型 & 是否读入地表臭氧浓度数据 \\
DEF\_USE\_SNICAR & 逻辑型 & 是否使用SNICAR积雪反照率模型 \\
DEF\_file\_snowoptics & 字符串 & 积雪光学属性参数数据文件 \\
DEF\_file\_snowaging  & 字符串 & 积雪雪龄(有效粒径)计算所需参数数据文件 \\
\multicolumn{3}{l}{DEF\_precip\_phase\_discrimination\_scheme} \\
(接上行)& 字符串 & 降水相态拆分方案:\textcolor{red}{待补充} \\
DEF\_SSP & 字符串 &  CMIP6未来情景二氧化碳路径 \\
\multicolumn{3}{l}{DEF\_USE\_Forcing\_Downscaling} \\ 
(接上行)& 字符串 & 是否对大气驱动进行降尺度 \\
\multicolumn{3}{l}{DEF\_DS\_precipitation\_adjust\_scheme} \\ 
(接上行)& 字符串 & 降水的降尺度调整方案:数值分别代表:I: Tesfa et al (2020); II: Liston and Elder (2006); III: 自动化机器学习方案 Chen et al (2024); \\
\multicolumn{3}{l}{DEF\_DS\_longwave\_adjust\_scheme} \\ 
(接上行)& 字符串 & 长波辐射的降尺度调整方案:数值分别代表: I: Fiddes and Gruber (2014); II: 假设长波辐射与海拔是线性变化关系 \\
DEF\_USE\_IRRIGATION & 逻辑型 & 是否使用灌溉模块 \\
DEF\_USE\_CBL\_HEIGHT & 逻辑型 & 是否使用大涡地表湍流方案LZD2022 \\
DEF\_USE\_PLANTHYDRAULICS & 逻辑型 & 是否使用植被水力模型 \\
DEF\_USE\_MEDLYNST & 逻辑型 & 是否使用MEDLYN气孔导度方案 \\
DEF\_USE\_SASU & 逻辑型 & 是否开启半解碳氮循环加速预热方案\\
DEF\_USE\_PN & 逻辑型 & 是否开启间断氮添加加速预热方案\\
DEF\_USE\_FERT & 逻辑型 & 是否开启作物施肥方案\\
DEF\_USE\_NITRIF & 逻辑型 & 是否开启硝化反硝化过程方案 \\
DEF\_USE\_CNSOYFIXN & 逻辑型 & 是否开启大豆固氮方案 \\
DEF\_USE\_FIRE & 逻辑型 & 是否开启火灾模型 \\

\end{longtable}


\subsection{城市模块}

当通过宏定义(\texttt {define.h})打开城市模式时,可以对其进行功能选项细分,包括所选用的城市次网格类型,模拟功能选项以及是否只模拟城市区域。次网格类型可以通过DEF\_URBAN\_type\_scheme设置,目前提供NCAR 3种城市分类和LCZ 10种城市分类两种方案。功能选项包括是否模拟城市中的植被、水体、建筑能耗、交通热及人体代谢热。如果只关注城市区域的模拟结果,可以通过定义DEF\_URBAN\_ONLY为true,此时只模拟计算网格中城市地表覆盖区域。

具体设置见表~\ref{table_nl_urban}。
\begin{table}[!htb]
\caption{Namelist变量:城市模块}
\label{table_nl_urban}
\centering \renewcommand{\arraystretch}{1.5}
\begin{tabular}{lcp{0.45\textwidth}}
\toprule
\textbf{变量名} & \textbf{数据类型} & \textbf{说明}\\\midrule
DEF\_URBAN\_type\_scheme & 整型 & 选项1: 采用NCAR 3种城市分类,即高建筑,高密度和中密度;选项2: 采用LCZ 10种城市分类,LCZ 1-10\\
DEF\_URBAN\_ONLY & 逻辑型 & 是否只模拟计算网格中城市地表覆盖区域,该选项可以忽略自然地表的模拟\\
DEF\_URBAN\_RUN & 逻辑型 & 根据宏定义预设,用户无需设置\\
DEF\_URBAN\_BEM & 逻辑型 & 是否打开城市建筑能耗模型\\
DEF\_URBAN\_TREE & 逻辑型 & 是否模拟城市中植被\\
DEF\_URBAN\_WATER & 逻辑型 & 是否模拟城市中水体\\
DEF\_URBAN\_LUCY & 逻辑型 & 是否打开LUCY模型模拟交通热、人体代谢热\\
\bottomrule
\end{tabular} 
\end{table}
\clearpage

\section{输入数据:高分辨率地表数据} \label{landdata}
\begin{longtable}{llp{0.75\textwidth}}
\caption{输入数据:高分辨率地表数据}\\

\toprule
\endfirsthead
\multicolumn{3}{r}{{续表}} \\
\toprule
\endhead
\midrule
\multicolumn{3}{r}{{续下页}} \\
\endfoot
\bottomrule
\endlastfoot

\multicolumn{3}{l}{\textbf{1. 地表覆盖类型}} \\
\midrule
\textbf{文件} & \multicolumn{2}{l}{landtypes/landtype-usgs-update.nc} \\
\cline{2-3}
&\textbf{变量名称} & landtype \\
&\textbf{变量维度} & {[}lon, lat{]}\\
&\textbf{变量含义} & 地表覆盖类型 \\
\cline{1-3}
\textbf{文件} & \multicolumn{2}{l}{landtypes/landtype-igbp-modis-\textit{yyyy}.nc} \\
&\textbf{说明} & IGBP地表覆盖提供多年数据,\textit{yyyy}表示年份 \\
\cline{2-3}
&\textbf{变量名称} & landtype \\
&\textbf{变量维度} & {[}lon, lat{]}\\
&\textbf{变量含义} & 地表覆盖类型; \\
\midrule
\vspace{2\baselineskip}\\
\multicolumn{3}{l}{\textbf{2. 植被功能类型覆盖百分比}} \\
\cline{1-3}
\textbf{文件} & \multicolumn{2}{l}{srf5x5/RG\_\textit{latn\_lonw\_lats\_lone}.MOD\textit{yyyy}.nc} \\
& \textbf{说明} & 分块存储,\textit{latn\_lonw\_lats\_lone}为分块存储信息,  \textit{lats}和\textit{latn}表示南北纬度边界,\textit{lone}和\textit{lonw}表示东西经度边界;\textit{yyyy}表示年份 \\
\cline{2-3}
 & \textbf{变量名称} & PCT\_PFT \\
 & \textbf{变量维度} & {[}lon, lat, pft{]} \\
 & \textbf{变量含义} & 植被功能类型覆盖百分比 \\

\midrule
\vspace{2\baselineskip}\\
 \multicolumn{3}{l}{\textbf{3. 树高}} \\
\cline{1-3}
\textbf{文件} & \multicolumn{2}{l}{Forest\_Height.nc} \\
\cline{2-3}
& \textbf{变量名称} & forest\_height \\
& \textbf{变量维度} & {[}lon, lat{]} \\

\midrule
\vspace{2\baselineskip}\\
\multicolumn{3}{l}{\textbf{4. 叶面积/茎面积指数}} \\
\cline{1-3}
\textbf{文件名} & \multicolumn{2}{l}{srf5x5/RG\_\textit{latn\_lonw\_lats\_lone}.MOD\textit{yyyy}.nc} \\
& \textbf{说明} & \textit{latn\_lonw\_lats\_lone}为分块存储信息,\textit{lats}和\textit{latn}表示南北纬度边界,\textit{lone}和\textit{lonw}表示东西经度边界;\textit{yyyy}为年份 \\
\cline{2-3}
& \textbf{变量名称} & MONTHLY\_LC\_LAI \\
& \textbf{变量维度} & {[}lon, lat, mon{]} \\
\cline{2-3}
& \textbf{变量名称} & MONTHLY\_LC\_SAI \\
& \textbf{变量维度} & {[}lon, lat, mon{]} \\
\cline{2-3}
& \textbf{变量名称} & MONTHLY\_PFT\_LAI \\
& \textbf{变量维度} & {[}lon, lat, mon{]} \\
\cline{2-3}
& \textbf{变量名称} & MONTHLY\_PFT\_SAI \\
& \textbf{变量维度} & {[}lon, lat, mon{]} \\
\cline{1-3}
\textbf{文件名} & \multicolumn{2}{l}{global\_lai\_15s\_release/lai\_8-day\_15s\_\textit{yyyy}.nc} \\
& \textbf{说明} & \textit{yyyy}为年份 \\
\cline{2-3}
& \textbf{变量名称} & lai \\
& \textbf{变量维度} & {[}lon, lat, time{]} \\

\midrule
\vspace{2\baselineskip}\\
\multicolumn{3}{l}{\textbf{5. 土壤参数数据}} \\
& \textbf{说明} & 以下土壤参数数据中的\textit{l}取值1-8,对应土壤8层垂直分层:(0-0.05m, 0.05-0.09m, 0.09-0.17m, 0.17-0.29m, 0.29-0.49m, 0.49-0.83m, 0.83-1.38m, 1.38-3.80m) \\
\cline{1-3}
\textbf{文件} & \multicolumn{2}{l}{soil/theta\_s.nc} \\
\cline{2-3}
& \textbf{变量名称} & theta\_s\_l\textit{l} \\
& \textbf{变量维度} & {[}lon, lat{]} \\
& \textbf{变量含义} & 饱和土壤含水量\\
\cline{1-3}
\textbf{文件} & \multicolumn{2}{l}{soil/k\_s.nc} \\
\cline{2-3}
& \textbf{变量名称} & k\_s\_l\textit{l} \\
& \textbf{变量维度} & {[}lon, lat{]} \\
& \textbf{变量含义} & 饱和土壤导水率 \\
\cline{1-3}
\textbf{文件} & \multicolumn{2}{l}{soil/psi\_s.nc} \\
\cline{2-3}
& \textbf{变量名称} &  psi\_s\_l\textit{l} \\
& \textbf{变量维度} & {[}lon, lat{]} \\
& \textbf{变量含义} & Campbell模型中的饱和基质势\\
\cline{1-3}
\textbf{文件} & \multicolumn{2}{l}{soil/lambda.nc} \\
\cline{2-3}
& \textbf{变量名称} &  lambda\_l\textit{l}\\
& \textbf{变量维度} & {[}lon, lat{]} \\
& \textbf{变量含义} & Campbell模型中的空隙大小分布指数\\
\cline{1-3}
\textbf{文件} & \multicolumn{2}{l}{soil/VGM\_alpha.nc} \\
\cline{2-3}
& \textbf{变量名称} & VGM\_alpha\_l\textit{l} \\
& \textbf{变量维度} & {[}lon, lat{]} \\
& \textbf{变量含义} & vanGenuchten\_Mualem模型中的形状参数\\
\cline{1-3}
\textbf{文件} & \multicolumn{2}{l}{soil/VGM\_n.nc} \\
\cline{2-3}
& \textbf{变量名称} &  VGM\_n\_l\textit{l}\\
& \textbf{变量维度} & {[}lon, lat{]} \\
& \textbf{变量含义} & vanGenuchten\_Mualem模型中的形状参数\\
\cline{1-3}
\textbf{文件} & \multicolumn{2}{l}{soil/VGM\_theta\_r.nc} \\
\cline{2-3}
& \textbf{变量名称} &  VGM\_theta\_r\_l\textit{l} \\
& \textbf{变量维度} & {[}lon, lat{]} \\
& \textbf{变量含义} & vanGenuchten\_Mualem模型中的残余土壤含水量 \\
\cline{1-3}
\textbf{文件} & \multicolumn{2}{l}{soil/VGM\_L.nc} \\
\cline{2-3}
& \textbf{变量名称} &  VGM\_L\_l\textit{l} \\
& \textbf{变量维度} & {[}lon, lat{]} \\
& \textbf{变量含义} & vanGenuchten\_Mualem模型中的孔隙导度参数\\
\cline{1-3}
\textbf{文件} & \multicolumn{2}{l}{soil/csol.nc} \\
\cline{2-3}
& \textbf{变量名称} & csol\_l\textit{l} \\
& \textbf{变量维度} & {[}lon, lat{]} \\
& \textbf{变量含义} & 固体土壤热容量 \\
\cline{1-3}
\textbf{文件} & \multicolumn{2}{l}{soil/k\_solids.nc} \\
\cline{2-3}
& \textbf{变量名称} & k\_solids\_l\textit{l} \\
& \textbf{变量维度} & {[}lon, lat{]} \\
& \textbf{变量含义} & 固体土壤导热率 \\
\cline{1-3}
\textbf{文件} & \multicolumn{2}{l}{soil/tkdry.nc} \\
\cline{2-3}
& \textbf{变量名称} & tkdry\_l\textit{l} \\
& \textbf{变量维度} & {[}lon, lat{]} \\
& \textbf{变量含义} & 干土壤导热率 \\
\cline{1-3}
\textbf{文件} & \multicolumn{2}{l}{soil/tksatf.nc} \\
\cline{2-3}
& \textbf{变量名称} &  tksatf\_l\textit{l} \\
& \textbf{变量维度} & {[}lon, lat{]} \\
& \textbf{变量含义} & 冻结状态饱和土壤导热率\\
\cline{1-3}
\textbf{文件} & \multicolumn{2}{l}{soil/tksatu.nc} \\
\cline{2-3}
& \textbf{变量名称} & tksatu\_l\textit{l} \\
& \textbf{变量维度} & {[}lon, lat{]} \\
& \textbf{变量含义} & 融化状态饱和土壤导热率 \\

\midrule
\vspace{2\baselineskip}\\
 \multicolumn{3}{l}{\textbf{6. 湖泊深度}} \\
\cline{1-3}
\textbf{文件} & \multicolumn{2}{l}{lake\_depth.nc} \\
\cline{2-3}
& \textbf{变量名称} & lake\_depth \\
& \textbf{变量维度} & {[}lon, lat{]} \\

\midrule
\vspace{2\baselineskip}\\
 \multicolumn{3}{l}{\textbf{7. 土壤亮度}} \\
\cline{1-3}
\textbf{文件} & \multicolumn{2}{l}{soil\_brightness.nc} \\
\cline{2-3}
& \textbf{变量名称} & soil\_brightness \\
& \textbf{变量维度} & {[}lon, lat{]} \\

\midrule
\vspace{2\baselineskip}\\
 \multicolumn{3}{l}{\textbf{8. 基岩深度}} \\
\cline{1-3}
\textbf{文件} & \multicolumn{2}{l}{dbedrock.nc} \\
\cline{2-3}
& \textbf{变量名称} & dbedrock \\
& \textbf{变量维度} & {[}lon, lat{]} \\

\midrule
\vspace{2\baselineskip}\\
\multicolumn{3}{l}{\textbf{9. 城市数据}} \\
\multicolumn{3}{l}{\textbf{9-1. 城市地表覆盖}} \\
\cline{1-3}
\textbf{文件} & \multicolumn{2}{l}{landtypes/landtype-usgs-update.nc} \\
\cline{2-3}
& \textbf{变量名称} & landtype \\
& \textbf{变量维度} & {[}lon, lat{]} \\
\cline{1-3}
\textbf{文件} & \multicolumn{2}{l}{landtypes/landtype-igbp-modis-\textit{yyyy}.nc}
\\
& \textbf{说明} & IGBP 13为城市地表,\textit{yyyy}表示年份 \\
\cline{2-3}
& \textbf{变量名称} & landtype \\
& \textbf{变量维度} & {[}lon, lat{]} \\

\midrule
\vspace{2\baselineskip}\\
\multicolumn{3}{l}{\textbf{9-2. 城市类型}} \\
\cline{1-3}
\textbf{文件} & \multicolumn{2}{l}{urban\_type/RG\_\textit{latn\_lonw\_lats\_lone}.URBTYP.nc} \\
& \textbf{说明} & 分块存储,\textit{latn\_lonw\_lats\_lone}为分块存储信息, \textit{lats}和\textit{latn}表示南北纬度边界,\textit{lone}和\textit{lonw}表示东西经度边界 \\
\cline{2-3}
& \textbf{变量名称} & URBAN\_DENSITY\_CLASS \\
& \textbf{变量维度} & {[}lon, lat{]} \\
\cline{2-3}
& \textbf{变量名称} & REGION\_ID \\
& \textbf{变量维度} & {[}lon, lat{]} \\
\cline{2-3}
& \textbf{变量名称} & LCZ\_DOM \\
& \textbf{变量维度} & {[}lon, lat{]} \\

\midrule
\vspace{2\baselineskip}\\
\multicolumn{3}{l}{\textbf{9-3. 城市叶面积/茎面积指数}} \\
\cline{1-3}
\textbf{文件} & \multicolumn{2}{l}{urban\_lai\_5x5/RG\_\textit{latn\_lonw\_lats\_lone}.UrbLAI\_\textit{yyyy}.nc} \\
& \textbf{说明} & 分块存储,\textit{latn\_lonw\_lats\_lone}为分块存储信息, \textit{lats}和\textit{latn}表示南北纬度边界,\textit{lone}和\textit{lonw}表示东西经度边界;\textit{yyyy}为年份 \\
\cline{2-3}
& \textbf{变量名称} & URBAN\_TREE\_LAI \\
& \textbf{变量维度} & {[}lon, lat, month{]} \\
\cline{2-3}
& \textbf{变量名称} & URBAN\_TREE\_SAI \\
& \textbf{变量维度} & {[}lon, lat, month{]} \\

\midrule
\vspace{2\baselineskip}\\
\multicolumn{3}{l}{\textbf{9-4. 城市树高}} \\
\cline{1-3}
\textbf{文件} & \multicolumn{2}{l}{urban/RG\_\textit{latn\_lonw\_lats\_lone}.URBSRF\textit{yyyy}.nc} \\
& \textbf{说明} & 分块存储,\textit{latn\_lonw\_lats\_lone}为分块存储信息, \textit{lats}和\textit{latn}表示南北纬度边界,\textit{lone}和\textit{lonw}表示东西经度边界;\textit{yyyy}为年份 \\
\cline{2-3}
& \textbf{变量名称} & HTOP \\
& \textbf{变量维度} & {[}lon, lat, month{]} \\

\midrule
\vspace{2\baselineskip}\\
\multicolumn{3}{l}{\textbf{9-5. 建筑高度/建筑比例}} \\
\cline{1-3}
\textbf{文件} & \multicolumn{2}{l}{urban/RG\_\textit{latn\_lonw\_lats\_lone}.URBSRF\textit{yyyy}.nc} \\
& \textbf{说明} & 分块存储,\textit{latn\_lonw\_lats\_lone}为分块存储信息, \textit{lats}和\textit{latn}表示南北纬度边界,\textit{lone}和\textit{lonw}表示东西经度边界;\textit{yyyy}为年份 \\
\cline{2-3}
& \textbf{变量名称} & HT\_ROOF \\
& \textbf{变量维度} & {[}lon, lat{]} \\
\cline{2-3}
& \textbf{变量名称} & PCT\_ROOF \\
& \textbf{变量维度} & {[}lon, lat{]} \\

\midrule
\vspace{2\baselineskip}\\
\multicolumn{3}{l}{\textbf{9-6. 植被/水体占比}} \\
\cline{1-3}
\textbf{文件} & \multicolumn{2}{l}{urban/RG\_\textit{latn\_lonw\_lats\_lone}.URBSRF\textit{yyyy}.nc} \\
& \textbf{说明} & 分块存储,\textit{latn\_lonw\_lats\_lone}为分块存储信息, \textit{lats}和\textit{latn}表示南北纬度边界,\textit{lone}和\textit{lonw}表示东西经度边界;\textit{yyyy}为年份 \\
\cline{2-3}
& \textbf{变量名称} & PCT\_Tree \\
& \textbf{变量维度} & {[}lon, lat{]} \\
\cline{2-3}
& \textbf{变量名称} & PCT\_Water \\
& \textbf{变量维度} & {[}lon, lat{]} \\

\midrule
\vspace{2\baselineskip}\\
\multicolumn{3}{l}{\textbf{9-7. 人口密度}} \\
\cline{1-3}
\textbf{文件} & \multicolumn{2}{l}{urban/RG\_\textit{latn\_lonw\_lats\_lone}.URBSRF\textit{yyyy}.nc} \\
& \textbf{说明} & 分块存储,\textit{latn\_lonw\_lats\_lone}为分块存储信息, \textit{lats}和\textit{latn}表示南北纬度边界,\textit{lone}和\textit{lonw}表示东西经度边界;\textit{yyyy}为年份 \\
\cline{2-3}
& \textbf{变量名称} & POP\_DEN \\
& \textbf{变量维度} & {[}lon, lat, year{]} \\

\midrule
\vspace{2\baselineskip}\\
\multicolumn{3}{l}{\textbf{9-8. LUCY国家分类}} \\
\cline{1-3}
\textbf{文件} & \multicolumn{2}{l}{urban/LUCY\_countryid.nc} \\
\cline{2-3}
& \textbf{变量名称} & LUCY\_COUNTRY\_ID \\
& \textbf{变量维度} & {[}lon, lat{]} \\

\end{longtable}

% \begin{longtable}{lll}
% % \centering \renewcommand{\arraystretch}{1.5}
% \caption{输入数据:高分辨率地表数据}\\
% \toprule
% \endfirsthead

% \multicolumn{3}{r}{{续表}} \\
% \toprule
% \endhead

% \midrule
% \multicolumn{3}{r}{{续下页}} \\
% \endfoot

% \bottomrule
% \endlastfoot

% \multicolumn{3}{l}{\textbf{1. 地表覆盖类型}} \\
% \textbf{文件名} & \multicolumn{2}{l}{\texttt{landtypes/landtype-usgs-update.nc}} \\
%  (包含变量)$\rightarrow$ & \textbf{变量名称} & \textbf{变量维度}\\
%  & landtype & {[}lon, lat{]} \\
%  \textbf{文件名} & \multicolumn{2}{l}{\texttt{landtypes/landtype-igbp-modis-yyyy.nc}} \\
%  (包含变量)$\rightarrow$ & \textbf{变量名称} & \textbf{变量维度}\\
%  & landtype & {[}lon, lat{]} \\
% \multicolumn{3}{l}{\colorbox{gray}{\textcolor{white}{\bf{注}}} IGBP地表覆盖提供多年数据,yyyy表示年份} \\
% \midrule
% \multicolumn{3}{l}{\textbf{2. 植被功能类型覆盖百分比}} \\
% \textbf{文件名} & \multicolumn{2}{l}{\texttt{srf5x5/RG\_latn\_lonw\_lats\_lone.MODyyyy.nc}} \\
% (包含变量)$\rightarrow$ & \textbf{变量名称} & \textbf{变量维度}\\
%  & PCT\_PFT & {[}lon, lat, pft{]} \\
% \multicolumn{3}{p{\textwidth}}{\colorbox{gray}{\textcolor{white}{\bf{注}}} 分块存储,latn\_lonw\_lats\_lone为分块存储信息, lats和latn表示南北纬度边界,lone和lonw表示东西经度边界} \\
% \midrule
% \multicolumn{3}{l}{\textbf{3. 树高}} \\
% \textbf{文件名} & \textbf{变量名称} & \textbf{变量维度}\\
% Forest\_Height.nc & forest\_height & {[}lon, lat{]} \\
% \midrule
% \multicolumn{3}{l}{\textbf{4. 叶面积/茎面积指数}} \\
% \textbf{文件名} & \multicolumn{2}{l}{\texttt{srf5x5/RG\_latn\_lonw\_lats\_lone.MODyyyy.nc}} \\
% (包含变量)$\rightarrow$ & \textbf{变量名称} & \textbf{变量维度} \\
% & MONTHLY\_LC\_LAI & {[}lon, lat, mon{]} \\
% & MONTHLY\_LC\_SAI & {[}lon, lat, mon{]} \\
% & MONTHLY\_PFT\_LAI & {[}lon, lat, pft, mon{]} \\
% & MONTHLY\_PFT\_SAI & {[}lon, lat, pft, mon{]} \\
% \textbf{文件名} & \multicolumn{2}{l}{\texttt{global\_lai\_15s\_release/lai\_8-day\_15s\_yyyy.nc}} \\
% (包含变量)$\rightarrow$ & \textbf{变量名称} & \textbf{变量维度} \\
% & lai & {[}lon, lat, time{]} \\
% \multicolumn{3}{l}{\colorbox{gray}{\textcolor{white}{\bf{注}}} latn\_lonw\_lats\_lone为分块存储信息,yyyy为年份,同上} \\

% \midrule
% \multicolumn{3}{l}{\textbf{5. 土壤参数数据}} \\
% \textbf{文件名} & \textbf{变量名称(含义)} & \textbf{变量维度}\\
% soil/theta\_s.nc & theta\_s\_l*(饱和土壤含水量) & {[}lon, lat{]} \\
% soil/k\_s.nc & \makecell[tl]{k\_s\_l* \\(饱和土壤导水率)} & {[}lon, lat{]} \\
% soil/psi\_s.nc & \makecell[tl]{psi\_s\_l*(Campbell模型中的\\饱和基质势)} & {[}lon, lat{]} \\
% soil/lambda.nc & \makecell[tl]{lambda\_l*(Campbell模型中的\\空隙大小分布指数)} & {[}lon, lat{]} \\
% soil/VGM\_alpha.nc & \makecell[tl]{VGM\_alpha\_l* \\(vGM$^+$模型中的形状参数)} & {[}lon, lat{]} \\
% soil/VGM\_n.nc & \makecell[tl]{VGM\_n\_l*\\ (vGM$^+$模型中的形状参数)} & {[}lon, lat{]} \\
% soil/VGM\_theta\_r.nc & \makecell[tl]{VGM\_theta\_r\_l* (vGM$^+$模型\\中的残余土壤含水量)} & {[}lon, lat{]} \\
% soil/VGM\_L.nc & \makecell[tl]{VGM\_L\_l* (vGM$^+$模型中的\\孔隙导度参数)} & {[}lon, lat{]} \\
% soil/csol.nc & csol\_l*(固体土壤热容量) & {[}lon, lat{]} \\
% soil/k\_solids.nc & k\_solids\_l*(固体土壤导热率) & {[}lon, lat{]} \\
% soil/tkdry.nc & tkdry\_l*(干土壤导热率) & {[}lon, lat{]} \\
% soil/tksatf.nc & \makecell[tl]{tksatf\_l*(冻结状态\\饱和土壤导热率)} & {[}lon, lat{]} \\
% soil/tksatu.nc & \makecell[tl]{tksatu\_l*(融化状态\\饱和土壤导热率)} & {[}lon, lat{]} \\
% \multicolumn{3}{p{0.9\textwidth}}{\colorbox{gray}{\textcolor{white}{\bf{注}}} *l取值1-8,对应土壤8层垂直分层:(0-0.05m, 0.05-0.09m, 0.09-0.17m, 0.17-0.29m, 0.29-0.49m, 0.49-0.83m, 0.83-1.38m, 1.38-3.80m)}  \\
% \multicolumn{3}{p{0.9\textwidth}}{$^+$vGM指vanGenuchten\_Mualem} \\

% \midrule
% \multicolumn{3}{l}{\textbf{6. 湖泊深度}} \\
% \textbf{文件名} & \textbf{变量名称} & \textbf{变量维度}\\
% lake\_depth.nc & lake\_depth & {[}lon, lat{]} \\

% \midrule
% \multicolumn{3}{l}{\textbf{7. 土壤亮度}} \\
% \textbf{文件名} & \textbf{变量名称} & \textbf{变量维度}\\\midrule
% soil\_brightness.nc & soil\_brightness & {[}lon, lat{]} \\

% \midrule
% \multicolumn{3}{l}{\textbf{8. 基岩深度}} \\
% \textbf{文件名} & \textbf{变量名称} & \textbf{变量维度}\\
% dbedrock.nc & dbedrock & {[}lon, lat{]} \\

% \midrule

% \multicolumn{3}{l}{\textbf{9. 城市数据}} \\
% \multicolumn{3}{l}{\textbf{9-1. 城市地表覆盖}} \\
% \textbf{文件名} & \multicolumn{2}{l}{\texttt{landtypes/landtype-usgs-update.nc}}\\
% (包含变量)$\rightarrow$ & \textbf{变量名称} & \textbf{变量维度}\\
%  & landtype & {[}lon, lat{]} \\
%  \textbf{文件名} & \multicolumn{2}{l}{\texttt{landtypes/landtype-igbp-modis-yyyy.nc}}\\
%  (包含变量)$\rightarrow$ & \textbf{变量名称} & \textbf{变量维度}\\
%  & landtype & {[}lon, lat{]} \\
% \multicolumn{3}{l}{\colorbox{gray}{\textcolor{white}{\bf{注}}} IGBP 13为城市地表,yyyy表示年份} \\

% \multicolumn{3}{l}{\textbf{9-2. 城市类型}} \\
% \textbf{文件名} & \multicolumn{2}{l}{\texttt{urban\_type/RG\_latn\_lonw\_lats\_lone.URBTYP.nc}} \\
% (包含变量)$\rightarrow$ & \textbf{变量名称} & \textbf{变量维度}\\
%  & URBAN\_DENSITY\_CLASS & {[}lon, lat{]} \\
% & REGION\_ID & {[}lon, lat{]} \\
% & LCZ\_DOM & {[}lon, lat{]} \\
% \multicolumn{3}{p{0.8\textwidth}}{\colorbox{gray}{\textcolor{white}{\bf{注}}} 分块存储,latn\_lonw\_lats\_lone为分块存储信息, lats和latn表示南北纬度边界,lone和lonw表示东西经度边界} \\

% \multicolumn{3}{l}{\textbf{9-3. 城市叶面积/茎面积指数}} \\
% \textbf{文件名} & \multicolumn{2}{l}{\texttt{urban\_lai\_5x5/RG\_latn\_lonw\_lats\_lone.UrbLAI\_yyyy.nc}}\\
% (包含变量)$\rightarrow$ & \textbf{变量名称} & \textbf{变量维度}\\
%  & URBAN\_TREE\_LAI & {[}lon, lat, month{]} \\
%  & URBAN\_TREE\_SAI & {[}lon, lat, month{]} \\
% \multicolumn{3}{p{0.8\textwidth}}{\colorbox{gray}{\textcolor{white}{\bf{注}}} latn\_lonw\_lats\_lone为分块存储信息,同上,yyyy为年份} \\

% \multicolumn{3}{l}{\textbf{9-4. 城市树高}} \\
% \textbf{文件名} & \multicolumn{2}{l}{\texttt{urban/RG\_latn\_lonw\_lats\_lone.URBSRFyyyy.nc}}\\
% (包含变量)$\rightarrow$ & \textbf{变量名称} & \textbf{变量维度}\\
%  & HTOP & {[}lon, lat, month{]} \\
% \multicolumn{3}{p{0.8\textwidth}}{\colorbox{gray}{\textcolor{white}{\bf{注}}} latn\_lonw\_lats\_lone为分块存储信息,yyyy为年份,同上} \\

% \multicolumn{3}{l}{\textbf{9-5. 建筑高度/建筑比例}} \\
% \textbf{文件名} & \multicolumn{2}{l}{\texttt{urban/RG\_latn\_lonw\_lats\_lone.URBSRFyyyy.nc}}\\
% (包含变量)$\rightarrow$  & \textbf{变量名称} & \textbf{变量维度}\\
%  & HT\_ROOF & {[}lon, lat{]} \\
%  & PCT\_ROOF & {[}lon, lat{]} \\
% \multicolumn{3}{p{0.8\textwidth}}{\colorbox{gray}{\textcolor{white}{\bf{注}}} latn\_lonw\_lats\_lone为分块存储信息,yyyy为年份,同上} \\

% \multicolumn{3}{l}{\textbf{9-6. 植被/水体占比}} \\
% \textbf{文件名} & \multicolumn{2}{l}{\texttt{urban/RG\_latn\_lonw\_lats\_lone.URBSRFyyyy.nc}} \\
% (包含变量)$\rightarrow$ & \textbf{变量名称} & \textbf{变量维度}\\
%  & PCT\_Tree & {[}lon, lat{]} \\
%  & PCT\_Water & {[}lon, lat{]} \\
% \multicolumn{3}{p{0.8\textwidth}}{\colorbox{gray}{\textcolor{white}{\bf{注}}} latn\_lonw\_lats\_lone为分块存储信息,yyyy为年份,同上} \\

% \multicolumn{3}{l}{\textbf{9-7. 人口密度}} \\
% \textbf{文件名} & \multicolumn{2}{l}{\texttt{urban/RG\_latn\_lonw\_lats\_lone.URBSRFyyyy.nc}}\\
% (包含变量)$\rightarrow$ & \textbf{变量名称} & \textbf{变量维度}\\
%  & POP\_DEN & {[}lon, lat, year{]} \\
% \multicolumn{3}{p{0.8\textwidth}}{\colorbox{gray}{\textcolor{white}{\bf{注}}} latn\_lonw\_lats\_lone为分块存储信息,yyyy为年份,同上} \\

% \multicolumn{3}{l}{\textbf{9-8. LUCY国家分类}} \\
% \textbf{文件名} & \textbf{变量名称} & \textbf{变量维度}\\
% \texttt{urban/LUCY\_countryid.nc} & LUCY\_COUNTRY\_ID & {[}lon, lat{]} \\

% \end{longtable}

%\clearpage

% \begin{table}[!htbp]
% \centering \renewcommand{\arraystretch}{1.5}
% \caption{高分辨率城市地表数据}
% \begin{tabular}{lll}
% \toprule

% \multicolumn{3}{l}{\textbf{1. 城市地表覆盖}} \\
% \textbf{文件名} & \textbf{变量名称} & \textbf{变量维度}\\
% landtypes/landtype-usgs-update.nc & landtype & {[}lon, lat{]} \\
% landtypes/landtype-igbp-modis-yyyy.nc & landtype & {[}lon, lat{]} \\
% \multicolumn{3}{l}{\colorbox{gray}{\textcolor{white}{\bf{注}}} IGBP 13为城市地表,yyyy表示年份} \\

% \multicolumn{3}{l}{\textbf{2. 城市类型}} \\
% \textbf{文件名} & \textbf{变量名称} & \textbf{变量维度}\\
% \multicolumn{3}{l}{urban\_type/} \\ 
% RG\_latn\_lonw\_lats\_lone.URBTYP.nc & URBAN\_DENSITY\_CLASS & {[}lon, lat{]} \\
% (文件同上)& REGION\_ID & {[}lon, lat{]} \\
% (文件同上)& LCZ\_DOM & {[}lon, lat{]} \\
% \multicolumn{3}{p{0.8\textwidth}}{\colorbox{gray}{\textcolor{white}{\bf{注}}} latn\_lonw\_lats\_lone为分块存储信息, lats和latn表示南北纬度边界,lone和lonw表示东西经度边界} \\

% \multicolumn{3}{l}{\textbf{3. 城市叶面积/茎面积指数}} \\
% \textbf{文件名} & \textbf{变量名称} & \textbf{变量维度}\\
% \multicolumn{3}{l}{urban\_lai\_5x5/} \\
% RG\_latn\_lonw\_lats\_lone.UrbLAI\_yyyy.nc & URBAN\_TREE\_LAI & {[}lon, lat, month{]} \\
% (文件同上)& URBAN\_TREE\_SAI & {[}lon, lat, month{]} \\
% \multicolumn{3}{p{0.8\textwidth}}{\colorbox{gray}{\textcolor{white}{\bf{注}}} latn\_lonw\_lats\_lone为分块存储信息,同上,yyyy为年份} \\

% \multicolumn{3}{l}{\textbf{4. 城市树高}} \\
% \textbf{文件名} & \textbf{变量名称} & \textbf{变量维度}\\
% \multicolumn{3}{l}{urban/} \\
% RG\_latn\_lonw\_lats\_lone.URBSRFyyyy.nc & HTOP & {[}lon, lat, month{]} \\
% \multicolumn{3}{p{0.8\textwidth}}{\colorbox{gray}{\textcolor{white}{\bf{注}}} latn\_lonw\_lats\_lone为分块存储信息,同上,yyyy为年份} \\

% \bottomrule
% \end{tabular}
% \end{table}
% \clearpage

\section{输入数据:大气驱动数据}

\begin{longtable}[!htbp]{lcp{0.4\textwidth}}
\caption{Namelist变量:大气驱动} \\

\toprule
\textbf{变量名} & \textbf{数据类型} & \textbf{说明} \\\midrule
\endfirsthead

\multicolumn{3}{r}
{{\bfseries \tablename\ \thetable{} -- \kaishu 续表}} \\
\toprule
\textbf{变量名} & \textbf{数据类型} & \textbf{说明} \\\midrule
\endhead

\midrule
\multicolumn{3}{r}{{续下页}} \\
\endfoot

\bottomrule
\endlastfoot

DEF\_forcing\_namelist & 字符串 & 驱动数据的namelist文件 \\
DEF\_dir\_forcing & 字符串 & 驱动数据的存放目录 \\
DEF\_Forcing & 自定义类型 & 定义驱动数据 \\
DEF\_Forcing\_Interp\_Method & 字符串 & 定义大气强迫场插值方案 \\
DEF\_Forcing\%dataset & 字符串 & 数据的名称 \\
DEF\_Forcing\%solarin\_all\_band & 逻辑型 & 下行短波辐射是否为全波段的 \\
DEF\_Forcing\%HEIGHT\_V & 浮点型 & 近地面风速场观测高度 \\
DEF\_Forcing\%HEIGHT\_T & 浮点型 & 近地面温度场观测高度 \\
DEF\_Forcing\%HEIGHT\_Q & 浮点型 & 近地面比湿场观测高度 \\
DEF\_Forcing\%regional & 逻辑型 & 是否是区域数据 \\
DEF\_Forcing\%regbnd & 浮点型数组 & 元素个数为4,数据为区域数据时,其南、北、西、东边界 \\
DEF\_Forcing\%has\_missing\_value & 逻辑型 & 是否有缺失值 \\ 
DEF\_Forcing\%missing\_value\_name & 字符串 & 有缺失值时,缺失值在数据文件中的变量名 \\ 
DEF\_Forcing\%NVAR & 整型 & 大气驱动变量个数(目前默认为8个) \\
DEF\_Forcing\%startyr & 整型 & 大气驱动资料的起始年份 \\
DEF\_Forcing\%startmo & 整型 & 大气驱动资料的起始月份 \\
DEF\_Forcing\%endyr & 整型 & 大气驱动资料的终止年份 \\
DEF\_Forcing\%endmo & 整型 & 大气驱动资料的终止月份 \\
DEF\_Forcing\%dtime(8) & 整型数组 & 8个大气驱动变量各自时间分辨率(秒) \\
DEF\_Forcing\%offset(8) & 整型数组 & 8个大气驱动变量各自第一个时刻距离上一个0时刻时间差值(秒) \\
DEF\_Forcing\%nlands & 整型 & 陆地格点的个数,仅在大气驱动数据是1维数据的情况下使用 \\
DEF\_Forcing\%leapyear & 逻辑型 & 大气驱动数据是否区分闰年 \\
DEF\_Forcing\%data2d & 逻辑型 & 大气驱动数据是否为2维经纬度网格数据 \\
DEF\_Forcing\%hightdim & 逻辑型 & 大气驱动数据是否包含高程维度 \\
DEF\_Forcing\%dim2d & 逻辑型 & 大气驱动数据的经纬度信息变量是否以2维经纬网格形式给出(否表示以一维形式给出) \\
DEF\_Forcing\%latname & 字符串 & 大气驱动数据的纬度信息变量名 \\
DEF\_Forcing\%lonname & 字符串 & 大气驱动数据的经度信息变量名 \\
DEF\_Forcing\%groupby & 字符串 & 大气驱动数据文件内数据的组织方式(如设为month表示每个文件存储一个月的数据) \\
DEF\_Forcing\%fprefix(8) & 字符串数组 & 相对DEF\_dir\_forcing的文件存储路径及文件名通用前缀。对数组中的8个变量(近地面温度、比湿、气压、降水率、纬向风速、经向风速、下行短波辐射、下行长波辐射)按顺序分别设置,缺失的变量设为NULL \\
DEF\_Forcing\%vname(8)  & 字符数组 & 大气驱动变量在文件中的变量名。按照温度(K)、比湿(\unit{kg.kg^{-1}})、气压(Pa)、降水率(\unit{mm.s^{-1}})、纬向风速(\unit{m.s^{-1}})、经向风速(\unit{m.s^{-1}})、下行短波辐射(\unit{W.m^{-2}})、下行长波辐射(\unit{W.m^{-2}})的顺序依次设置,缺失的变量设为NULL \\
DEF\_Forcing\%tintalgo(8) & 字符串数组 & 大气驱动变量在模式中的时间维插值方案。按照温度、比湿、气压、降水率、纬向风速、经向风速、下行短波辐射、下行长波辐射的顺序依次设置。目前默认的方案为温度、比湿、气压、风速和下行长波辐射采用线性插值方案,降水率采用临近时刻插值方案,下行短波辐射采用根据太阳高度角的余弦插值方案。缺失的变量设为NULL \\
DEF\_Forcing\%CBL\_fprefix & 字符串 & fprefix同上,对于大气边界层高度 \\
DEF\_Forcing\%CBL\_vname & 字符串 & vname同上,对于大气边界层高度 \\
DEF\_Forcing\%CBL\_tintalgo & 字符串 & tintalgo同上,对于大气边界层高度 \\
DEF\_Forcing\%CBL\_dtime & 整型 & dtime同上,对于大气边界层高度 \\
DEF\_Forcing\%CBL\_offset & 整型 & offset同上,对于大气边界层高度 \\
DEF\_USE\_CBL\_HEIGHT & 逻辑型 & 是否读入大气边界层高度数据 \\
\end{longtable}

\begin{table}[!htbp]
\caption{可使用的大气驱动数据}
\centering \renewcommand{\arraystretch}{1.5}
\begin{tabular}[]{cccc}
\toprule
\textbf{数据名称} & \textbf{分辨率} & \textbf{时间跨度} & \textbf{说明} \\\midrule

QIAN & 1.875°/6-hourly & 1948-2004 & Global \\
CRU-NCEP\_V4 & 0.5°/6-hourly & 1980-2014 & Global \\
CRU-NCEP\_V7 & 0.5°/6-hourly & 1901-2016 & Global \\
CRUJRA2.2 & 0.5°/6-hourly & 1901-2022 & Global \\
Princeton & 0.5°/3-hourly & 1901-2012 & Global \\
GSWP3 & 0.5°/3-hourly & 1901-2014 & Global \\
GDAS\_GPCP & 0.5°/3-hourly & 2002-2021 & Global (land only) \\
WFDEI & 0.5°/3-hourly & 1979-2016 & Global (land only) \\
ERA5 & 0.25°/hourly & 1950-2021 & Global \\
ERA5LAND & 0.1°/hourly & 1951-2021 & Global (land only) \\
MSWX\_V100 & 0.1°/3-hourly & 1979-2021 & Global \\
JRA55 & 0.5625°/3-hourly & 1979-2022 & Global \\
WFDE5 & 0.5°/hourly & 1979-2019 & Global (land only) \\
CLDAS & 0.0625°/hourly & 2008-2020 &
亚洲区域(60°\textasciitilde160°E, 0°\textasciitilde65°N) \\
CMFD & 0.1°/3-hourly & 1979-2018 & 中国区域
(70°\textasciitilde140°E, 15°\textasciitilde55°N) \\
TPMFD & 0.0333°/3-hourly & 1979-2021 & \makecell{青藏高原
(61.0°\textasciitilde105.678°E, \\
25.4166°\textasciitilde41.3818°N)} \\
\bottomrule
\end{tabular}
\end{table}

\section{输出数据:重启动文件}\label{restart}

\begin{table}[!htbp]
\caption{输出数据:重启动文件}
\centering \renewcommand{\arraystretch}{1.2}
\begin{tabular}{lcp{0.4\textwidth}}
\toprule
\textbf{变量名} & \textbf{数据类型} & \textbf{说明} \\\midrule
DEF\_REST\_FREQ & 字符串 & 重启动文件输出的频率,可选HOURLY, DAILY, MONTHLY, YEARLY \\
DEF\_REST\_COMPRESS\_LEVEL & 整型 & 重启动文件的压缩级别,可选从0到10的整数,数字越大代表压缩比越高;对于高压缩比,压缩或者解压所需的时间也会更长 \\
\bottomrule
\end{tabular} 
\end{table}

\section{输出数据:历史文件}\label{history}

\subsection{namelist文件中的设置}
\begin{longtable}[htbp]{lcp{0.42\textwidth}}
\caption{输出数据:历史文件} \\
\toprule
\textbf{变量名} & \textbf{数据类型} & \textbf{说明} \\\midrule
\endfirsthead
\multicolumn{3}{r}
{{\bfseries \tablename\ \thetable{} -- \kaishu 续表}} \\
\toprule
\textbf{变量名} & \textbf{数据类型} & \textbf{说明}  \\\midrule
\endhead
\midrule
\multicolumn{3}{r}{{续下页}} \\
\endfoot
\bottomrule
\endlastfoot

DEF\_HISTORY\_IN\_VECTOR & 逻辑型 & 对\textbf{流域单元网格}和\textbf{非结构网格},以向量的形式输出历史文件(TRUE),或者以经纬度格点数据的形式输出历史文件(FALSE) \\
DEF\_HIST\_grid\_as\_forcing & 逻辑型 & 以经纬度格点数据的形式输出历史文件时,是否使用与驱动数据相同的网格 \\
DEF\_HIST\_lon\_res & 浮点型 & 以经纬度格点数据的形式输出历史文件且不使用与驱动数据相同的网格时,输出变量的经度方向分辨率 \\
DEF\_HIST\_lat\_res & 浮点型 & 以经纬度格点数据的形式输出历史文件与不使用与驱动数据相同的网格时,输出变量的纬度方向分辨率 \\
DEF\_HIST\_FREQ & 字符串 & 变量历史输出的频率,可选HOURLY, DAILY, MONTHLY, YEARLY \\
DEF\_HIST\_groupby & 字符串 & 变量历史数据在文件中的组织形式,可选DAY, MONTH, YEAR,分别表示每天一个文件、每月一个文件或每年一个文件 \\
DEF\_HIST\_mode & 字符串 & 数据文件的组织形式,值为~'one'时,模式将模拟区域的数据拼接完整后输出,值为~'block'时,模式将每块数据输出到单独的文件中\\
DEF\_HIST\_COMPRESS\_LEVEL & 整型 & 变量历史文件的压缩级别,可选从0到10的整数,数字越大代表压缩比越高;对于高压缩比,压缩或者解压所需的时间也会更长 \\
DEF\_HIST\_vars & 自定义类型 & 其元素为定义是否输出变量的开关,元素名为输出变量的名字,元素的数据类型为逻辑性,真值代表输出,假值代表不输出;变量列表见第\ref{sec_hist_vars_basic}-\ref{sec_hist_vars_crop}节 \\
DEF\_HIST\_vars\_out\_default & 逻辑型 & 所有变量开关的默认值,值为TRUE时全部输出,值为FALSE时全部不输出;默认值会被文件DEF\_HIST\_vars\_namelist中的变量开关的值修改\\
DEF\_HIST\_vars\_namelist & 字符串 & 变量开关的namelist文件;输出变量开关的列表较长时,可将其放置在此独立的文件中,namelist的名称为nl\_colm\_history,文件中的开关值会覆盖由DEF\_HIST\_vars\_out\_default定义的默认值 \\
\end{longtable}
\clearpage

\subsection{基本输出变量} \label{sec_hist_vars_basic}


\begin{longtable}[htbp]{lp{0.3\textwidth}p{0.22\textwidth}l}
\caption[基本输出变量]{基本输出变量} \\

\toprule
\textbf{变量名} & \textbf{说明} & \textbf{维数} & \textbf{单位} \\\midrule
\endfirsthead

\multicolumn{4}{r}
{{\bfseries \tablename\ \thetable{} -- \kaishu 续表}} \\
\toprule
\textbf{变量名} & \textbf{说明} & \textbf{维数} & \textbf{单位} \\\midrule
\endhead

\midrule
\multicolumn{3}{r}{{续下页}} \\
\endfoot
\bottomrule
\endlastfoot

xy\_us & 经向风速 & {[}lon, lat, time{]} & \unit{m.s^{-1}} \\
xy\_vs & 纬向风速 & {[}lon, lat, time{]} & \unit{m.s^{-1}} \\
xy\_t & 参考高度气温 & {[}lon, lat, time{]}  & K \\
xy\_q & 参考高度比湿 & {[}lon, lat, time{]}  & \unit{kg.kg^{-1}} \\
xy\_prc & 对流降水 & {[}lon, lat, time{]}  & \unit{mm.s^{-1}} \\
xy\_prl & 大尺度降水 & {[}lon, lat, time{]}  & \unit{mm.s^{-1}} \\
xy\_pbot & 地表气压 & {[}lon, lat, time{]}  & Pa \\
xy\_frl & 大气向下长波辐射 & {[}lon, lat, time{]}  & \unit{W.m^{-2}} \\
xy\_solarin & 下行太阳短波辐射 & {[}lon, lat, time{]}  & \unit{W.m^{-2}} \\
xy\_rain & 降水 & {[}lon, lat, time{]}  & \unit{mm.s^{-1}} \\
xy\_snow & 降雪 & {[}lon, lat, time{]}  & \unit{mm.s^{-1}} \\
taux & 经向风应力 & {[}lon, lat, time{]}  & \unit{kg.m^{-1}.s^{-2}} \\
tauy & 纬向风应力 & {[}lon, lat, time{]}  & \unit{kg.m^{-1}.s^{-2}} \\
fsena & 从冠层到大气的感热 & {[}lon, lat, time{]}  & \unit{W.m^{-2}} \\
lfevpa & 从冠层到大气的潜热 & {[}lon, lat, time{]}  & \unit{W.m^{-2}} \\
fevpa & 从冠层到大气的蒸散发 & {[}lon, lat, time{]}  & \unit{mm.s^{-1}} \\
fsenl & 叶片感热 & {[}lon, lat, time{]}  & \unit{W.m^{-2}} \\
fevpl & 叶片蒸散发 & {[}lon, lat, time{]}  & \unit{mm.s^{-1}} \\
etr & 蒸腾速率 & {[}lon, lat, time{]}  & \unit{mm.s^{-1}} \\
fseng & 地面感热 & {[}lon, lat, time{]}  & \unit{W.m^{-2}} \\
fevpg & 地面蒸发 & {[}lon, lat, time{]}  & \unit{mm.s^{-1}} \\
fgrnd & 地表热通量 & {[}lon, lat, time{]}  & \unit{W.m^{-2}} \\
sabvsun & 阳叶吸收的太阳辐射 & {[}lon, lat, time{]}  & \unit{W.m^{-2}} \\
sabvsha & 阴叶吸收的太阳辐射 & {[}lon, lat, time{]}  & \unit{W.m^{-2}} \\
sabg & 地面吸收的太阳辐射 & {[}lon, lat, time{]}  & \unit{W.m^{-2}} \\
olrg & 地表向上发射长波辐射 & {[}lon, lat, time{]}  & \unit{W.m^{-2}} \\
rnet & 净辐射 & {[}lon, lat, time{]}  & \unit{W.m^{-2}} \\
xerr & 水量平衡误差 & {[}lon, lat, time{]}  & \unit{mm.s^{-1}} \\
zerr & 能量平衡误差 & {[}lon, lat, time{]}  & \unit{W.m^{-2}} \\
rsur & 地表径流 & {[}lon, lat, time{]}  & \unit{mm.s^{-1}} \\
rnof & 总径流 & {[}lon, lat, time{]}  & \unit{mm.s^{-1}} \\
qintr & 植被截留水通量 & {[}lon, lat, time{]}  & \unit{mm.s^{-1}} \\
qinfl & 土壤入渗水通量 & {[}lon, lat, time{]}  & \unit{mm.s^{-1}} \\
qdrip & 植被穿透水通量 & {[}lon, lat, time{]}  & \unit{mm.s^{-1}} \\
wat & 总水储量 & {[}lon, lat, time{]}  & mm \\
assim & 冠层同化速率 & {[}lon, lat, time{]}  & \unit{mol.m^{-2}.s^{-1}} \\
respc & 叶呼吸 & {[}lon, lat, time{]}  & \unit{mol.m^{-2}.s^{-1}} \\
qcharge & 地下水补给速率 & {[}lon, lat, time{]}  & \unit{mm.s^{-1}} \\
t\_grnd & 地面温度 & {[}lon, lat, time{]}  & K \\
tleaf & 叶片温度 & {[}lon, lat, time{]}  & K \\
ldew & 叶片载水量 & {[}lon, lat, time{]}  & mm \\
scv & 雪水当量 & {[}lon, lat, time{]}  & mm \\
snowdp & 雪深 & {[}lon, lat, time{]}  & m \\
fsno & 地面雪覆盖百分比 & {[}lon, lat, time{]}  & - \\
sigf & 除去雪覆盖/掩埋后的植被覆盖百分比 & {[}lon, lat, time{]}  & - \\
green & 叶片绿度 & {[}lon, lat, time{]}  & - \\
lai & 叶面积指数 & {[}lon, lat, time{]}  & - \\
laisun & 阳叶叶面积指数 & {[}lon, lat, time{]}  & - \\
laisha & 阴叶叶面积指数 & {[}lon, lat, time{]}  & - \\
sai & 茎面积指数 & {[}lon, lat, time{]}  & - \\
alb & 平均反照率 & {[}band, rad, lon, lat, time{]}  & - \\ % radiation type
emis & 等效地表发射率 & {[}lon, lat, time{]}  & - \\
z0m & 等效地表粗糙度 & {[}lon, lat, time{]}  & m \\
trad & 地表辐射温度 & {[}lon, lat, time{]}  & K \\
tref & 地表2米气温 & {[}lon, lat, time{]}  & K \\
qref & 地表2米空气比湿 & {[}lon, lat, time{]}  & \unit{kg.kg^{-1}} \\
t\_soisno ~ ~ ~ & 土壤温度 & {[}soil layer, lon, lat, time{]} & K \\
wliq\_soisno~ ~ & 土壤层中的液态水含量 & {[}soil layer, lon, lat, time{]} & \unit{kg.m^{-2}} \\
wice\_soisno~ ~ & 土壤层中的固态水含量& {[}soil layer, lon, lat, time{]} & \unit{kg.m^{-2}} \\
h2osoi ~ ~ ~ ~ ~ ~ ~ & 土壤层中的体积含水量 & {[}soil layer, lon, lat,
time{]}  & \unit{m^3.m^{-3}} \\
rstfacsun ~ ~ & 阳叶土壤水胁迫因子 & {[}lon, lat, time{]}  & - \\
rstfacsha ~ ~ & 阴叶土壤水胁迫因子 & {[}lon, lat, time{]}  & - \\
rootr ~ ~ ~ ~ ~ ~ ~ & 土壤与根系交换的水量,各层之和为1 & {[}soil layer, lon, lat, time{]}  & - \\
vegwp & 植物水势 & {[}vegnode, lon, lat, time{]}  & mm \\ % vegetation water potential node
dpond & 地表积水深度 & {[}lon, lat, time{]}  & mm \\
dwatsub & 基岩之上的饱和层厚度 & {[}lon, lat, time{]}  & m \\
zwt & 地下水位 & {[}lon, lat, time{]}  & m \\
wa & 蓄水层储水量 & {[}lon, lat, time{]}  & mm \\
t\_lake & 湖泊温度 & {[}lake layer, lon, lat, time{]}  & K \\
lake\_icefrac & 湖泊冰百分比 & {[}lake layer, lon, lat, time{]} & - \\
ustar~ ~ & 相似性理论中的$u^*$ & {[}lon, lat, time{]}  & \unit{m.s^{-1}} \\
tstar~ ~ & 相似性理论中的$\theta ^*$ & {[}lon, lat, time{]}  & K \\
qstar~ ~ & 相似性理论中的$q^*$ & {[}lon, lat, time{]}  & \unit{kg.kg^{-1}} \\
zol~ ~ ~ & Monin-Obukhov理论中的无量纲高度$\zeta$ & {[}lon, lat, time{]} & - \\
rib~ ~ ~ & 地表的体积Richardson数 & {[}lon, lat, time{]}  & - \\
fm ~ ~ ~ & 通量的剖面函数积分 & {[}lon, lat, time{]}  & - \\
fh ~ ~ ~ & 热量的剖面函数积分 & {[}lon, lat, time{]}  & - \\
fq ~ ~ ~ & 湿度的剖面函数积分 & {[}lon, lat, time{]}  & - \\
us10m~ ~ & 10米经向风速 & {[}lon, lat, time{]}  & \unit{m.s^{-1}} \\
vs10m~ ~ & 10米纬向风速 & {[}lon, lat, time{]}  & \unit{m.s^{-1}} \\
fm10m~ ~ & 10米通量的剖面函数积分 & {[}lon, lat, time{]}  & - \\
sr ~ ~ ~ & 地表反射太阳辐射 & {[}lon, lat, time{]}  & \unit{W.m^{-2}} \\
solvd~ ~ & 可见光入射直射太阳辐射 & {[}lon, lat, time{]}  & \unit{W.m^{-2}} \\
solvi~ ~ & 可见光漫射入射太阳辐射 & {[}lon, lat, time{]}  & \unit{W.m^{-2}} \\
solnd~ ~ & 近红外直射入射太阳辐射 & {[}lon, lat, time{]}  & \unit{W.m^{-2}} \\
solni~ ~ & 近红外漫射入射太阳辐射 & {[}lon, lat, time{]}  & \unit{W.m^{-2}} \\
srvd ~ ~ & 地表反射的可见光直射入射太阳辐射 & {[}lon, lat, time{]}  & \unit{W.m^{-2}} \\
srvi ~ ~ & 地表反射的可见光漫射入射太阳辐射 & {[}lon, lat, time{]}  & \unit{W.m^{-2}} \\
srnd ~ & 地表反射的近红外直射入射太阳辐射 & {[}lon, lat, time{]}  & \unit{W.m^{-2}} \\
srni ~ & 地表反射的近红外漫射入射太阳辐射 & {[}lon, lat, time{]}  & \unit{W.m^{-2}} \\
solvdln~ & 当地正午时间,可见光直射入射太阳辐射 & {[}lon, lat,
time{]}  & \unit{W.m^{-2}} \\
solviln~ & 当地正午时间,可见光漫射入射太阳辐射 & {[}lon, lat,
time{]}  & \unit{W.m^{-2}} \\
solndln~ & 当地正午时间,近红外直射入射太阳辐射 & {[}lon, lat,
time{]}  & \unit{W.m^{-2}} \\
solniln~ & 当地正午时间,近红外漫射入射太阳辐射 & {[}lon, lat,
time{]}  & \unit{W.m^{-2}} \\
srvdln ~ & 当地正午时间,地表反射的可见光直射入射太阳辐射 & {[}lon, lat,
time{]}  & \unit{W.m^{-2}} \\
srviln ~ & 当地正午时间,地表反射的可见光漫射入射太阳辐射 & {[}lon, lat,
time{]}  & \unit{W.m^{-2}} \\
srndln ~ & 当地正午时间,地表反射的近红外直射入射太阳辐射 & {[}lon, lat,
time{]}  & \unit{W.m^{-2}} \\
srniln~ & 当地正午时间,地表反射的近红外漫射入射太阳辐射 & {[}lon, lat,
time{]}  & \unit{W.m^{-2}} \\

\end{longtable}

\subsection{城市模块输出变量}
\begin{longtable}[htbp]{lp{0.3\textwidth}ll}
\caption{城市模块输出变量} \\

\toprule
\textbf{变量名} & \textbf{说明} & \textbf{维数} & \textbf{单位} \\\midrule
\endfirsthead

\multicolumn{4}{r}
{{\bfseries \tablename\ \thetable{} -- \kaishu 续表}} \\
\toprule
\textbf{变量名} & \textbf{说明} & \textbf{维数} & \textbf{单位} \\\midrule
\endhead

\bottomrule
\endfoot
\bottomrule
\endlastfoot

fsen\_roof & 屋顶感热 & {[}lon, lat, time{]} & \unit{W.m^{-2}} \\
fsen\_wsun & 阳面墙感热 & {[}lon, lat, time{]} & \unit{W.m^{-2}} \\
fsen\_wsha & 阴面墙感热 & {[}lon, lat, time{]} & \unit{W.m^{-2}} \\
fsen\_gimp & 不透水面感热 & {[}lon, lat, time{]} & \unit{W.m^{-2}} \\
fsen\_gper & 透水面感热 & {[}lon, lat, time{]} & \unit{W.m^{-2}} \\
fsen\_urbl & 城市植被感热 & {[}lon, lat, time{]} & \unit{W.m^{-2}} \\
lfevp\_roof & 屋顶潜热 & {[}lon, lat, time{]} & \unit{W.m^{-2}} \\
lfevp\_gimp & 不透水面潜热 & {[}lon, lat, time{]} & \unit{W.m^{-2}} \\
lfevp\_gper & 透水面潜热 & {[}lon, lat, time{]} & \unit{W.m^{-2}} \\
lfevp\_urbl & 城市植被潜热 & {[}lon, lat, time{]} & \unit{W.m^{-2}} \\
fhac & 空调制冷通量 & {[}lon, lat, time{]} & \unit{W.m^{-2}} \\
fach & 室内外空气交换通量 & {[}lon, lat, time{]} & \unit{W.m^{-2}} \\
fhah & 空调加热通量 & {[}lon, lat, time{]} & \unit{W.m^{-2}} \\
fwst & 空调排放废热 & {[}lon, lat, time{]} & \unit{W.m^{-2}} \\
vehc & 交通热通量 & {[}lon, lat, time{]} & \unit{W.m^{-2}} \\
meta & 代谢热通量 & {[}lon, lat, time{]} & \unit{W.m^{-2}} \\
t\_room & 室内温度 & {[}lon, lat, time{]} & K \\
t\_roof & 屋顶温度 & {[}lon, lat, time{]} & K \\
t\_wall & 墙体温度 & {[}lon, lat, time{]} & K \\
tafu & 室外温度 & {[}lon, lat, time{]} & K \\
\end{longtable}


\subsection{生物地球化学循环模块输出变量}
\begin{longtable}[htbp]{lp{0.3\textwidth}ll}
\caption{生物地球化学循环模块输出变量} \\

\toprule
\textbf{变量名} & \textbf{说明} & \textbf{维数} & \textbf{单位} \\\midrule
\endfirsthead

\multicolumn{4}{r}
{{\bfseries \tablename\ \thetable{} -- \kaishu 续表}} \\
\toprule
\textbf{变量名} & \textbf{说明} & \textbf{维数} & \textbf{单位} \\\midrule
\endhead

\bottomrule
\endfoot
\bottomrule
\endlastfoot
leafc & 叶碳库 & {[}lon, lat, time{]} & gCm\textsuperscript{-2} \\
leafc\_storage & 叶储存碳库 & {[}lon, lat, time{]}  & gCm\textsuperscript{-2} \\
leafc\_xfer & 叶传输碳库 & {[}lon, lat, time{]}  & gCm\textsuperscript{-2} \\
frootc & 细根碳库 & {[}lon, lat, time{]}  & gCm\textsuperscript{-2} \\
frootc\_storage& 细根储存碳库 & {[}lon, lat, time{]} & gCm\textsuperscript{-2} \\
frootc\_xfer & 细根传输碳库 & {[}lon, lat, time{]} & gCm\textsuperscript{-2} \\
livestemc & 活茎碳库 & {[}lon, lat, time{]}  & gCm\textsuperscript{-2} \\
livestemc\_storage & 活茎储存碳库 & {[}lon, lat, time{]} & gCm\textsuperscript{-2} \\
livestemc\_xfer & 活茎传输碳库 & {[}lon, lat, time{]} & gCm\textsuperscript{-2} \\
deadstemc & 死茎碳库 & {[}lon, lat, time{]}  & gCm\textsuperscript{-2} \\
deadstemc\_storage  & 死茎储存碳库 & {[}lon, lat, time{]} & gCm\textsuperscript{-2} \\
deadstemc\_xfer & 死茎传输碳库 & {[}lon, lat, time{]} & gCm\textsuperscript{-2} \\
livecrootc & 活粗根碳库 & {[}lon, lat, time{]}  & gCm\textsuperscript{-2} \\
livecrootc\_storage & 活粗根储存碳库 & {[}lon, lat, time{]} & gCm\textsuperscript{-2} \\
livecrootc\_xfer   & 活粗根传输碳库 & {[}lon, lat, time{]} & gCm\textsuperscript{-2} \\
deadcrootc     & 死粗根碳库 & {[}lon, lat, time{]}  & gCm\textsuperscript{-2} \\
deadcrootc\_storage & 死粗根储存碳库 & {[}lon, lat, time{]}  & gCm\textsuperscript{-2} \\
deadcrootc\_xfer   & 死粗根传输碳库 & {[}lon, lat, time{]} & gCm\textsuperscript{-2} \\
grainc       & 谷粒碳库 & {[}lon, lat, time{]}  & gCm\textsuperscript{-2} \\
grainc\_storage   & 谷粒储存碳库 & {[}lon, lat, time{]} & gCm\textsuperscript{-2} \\
grainc\_xfer     & 谷粒传输碳库 & {[}lon, lat, time{]} & gCm\textsuperscript{-2} \\
leafn        & 叶氮库 & {[}lon, lat, time{]}  & gNm\textsuperscript{-2} \\
leafn\_storage    & 叶储存氮库 & {[}lon, lat, time{]}  & gNm\textsuperscript{-2} \\
leafn\_xfer     & 叶传输氮库 & {[}lon, lat, time{]}  & gNm\textsuperscript{-2} \\
frootn       & 细根氮库 & {[}lon, lat, time{]}  & gNm\textsuperscript{-2} \\
frootn\_storage   & 细根储存氮库 & {[}lon, lat, time{]} & gNm\textsuperscript{-2} \\
frootn\_xfer     & 细根传输氮库 & {[}lon, lat, time{]} & gNm\textsuperscript{-2} \\
livestemn      & 活茎氮库 & {[}lon, lat, time{]}  & gNm\textsuperscript{-2} \\
livestemn\_storage  & 活茎储存氮库 & {[}lon, lat, time{]} & gNm\textsuperscript{-2} \\
livestemn\_xfer   & 活茎传输氮库 & {[}lon, lat, time{]} & gNm\textsuperscript{-2} \\
deadstemn      & 死茎氮库 & {[}lon, lat, time{]}  & gNm\textsuperscript{-2} \\
deadstemn\_storage  & 死茎储存氮库 & {[}lon, lat, time{]} & gNm\textsuperscript{-2} \\
deadstemn\_xfer   & 死茎传输氮库 & {[}lon, lat, time{]} & gNm\textsuperscript{-2} \\
livecrootn & 活粗根氮库 & {[}lon, lat, time{]}  & gNm\textsuperscript{-2} \\
livecrootn\_storage & 活粗根储存氮库 & {[}lon, lat, time{]} & gNm\textsuperscript{-2} \\
livecrootn\_xfer & 活粗根传输氮库 & {[}lon, lat, time{]}  & gNm\textsuperscript{-2} \\
deadcrootn & 死粗根氮库 & {[}lon, lat, time{]}  & gNm\textsuperscript{-2} \\
deadcrootn\_storage & 死粗根储存氮库 & {[}lon, lat, time{]} & gNm\textsuperscript{-2} \\
deadcrootn\_xfer & 死粗根传输氮库 & {[}lon, lat, time{]}  & gNm\textsuperscript{-2} \\
grainn & 谷粒氮库 & {[}lon, lat, time{]}  & gNm\textsuperscript{-2} \\
grainn\_storage & 谷粒储存氮库 & {[}lon, lat, time{]}  & gNm\textsuperscript{-2} \\
grainn\_xfer & 谷粒传输氮库 & {[}lon, lat, time{]}  & gNm\textsuperscript{-2} \\
retrasn & 再利用氮库 & {[}lon, lat, time{]}  & gNm\textsuperscript{-2} \\
gpp & 总第一性生产力 & {[}lon, lat, time{]}  & gCm\textsuperscript{-2}s\textsuperscript{-1} \\
downreg & 氮限制因子 & {[}lon, lat, time{]}  & - \\
ar & 自养呼吸速率 & {[}lon, lat, time{]}  & gCm\textsuperscript{-2}s\textsuperscript{-1} \\
hr & 异养呼吸速率 &  {[}lon, lat, time{]}  & gCm\textsuperscript{-2}s\textsuperscript{-1} \\
cwdprod & 粗木质残体产生速率 &  {[}lon, lat, time{]}  & gCm\textsuperscript{-2}s\textsuperscript{-1} \\
cwddecomp & 粗木质残体分解速率 &  {[}lon, lat, time{]}  & gCm\textsuperscript{-2}s\textsuperscript{-1} \\
cwddecomp & 粗木质残体分解速率 &  {[}lon, lat, time{]}  & gCm\textsuperscript{-2}s\textsuperscript{-1} \\
fpg & 植被生产力氮限制因子 &  {[}lon, lat, time{]}  & - \\
fpi & 土壤分解氮限制因子 &  {[}lon, lat, time{]}  & - \\
litr1c\_vr & 代谢凋落物碳 & {[}soil, lon, lat, time{]}  & gCm\textsuperscript{-3} \\
litr2c\_vr & 纤维素凋落物碳 & {[}soil, lon, lat, time{]}  & gCm\textsuperscript{-3} \\
litr3c\_vr & 木质素凋落物碳 & {[}soil, lon, lat, time{]}  & gCm\textsuperscript{-3} \\
soil1c\_vr & 快速周转土壤碳 & {[}soil, lon, lat, time{]}  & gCm\textsuperscript{-3} \\
soil2c\_vr & 慢速周转土壤碳 & {[}soil, lon, lat, time{]}  & gCm\textsuperscript{-3} \\
soil3c\_vr & 惰性土壤碳 & {[}soil, lon, lat, time{]}  & gCm\textsuperscript{-3} \\
cwdc\_vr & 粗木质残体碳 & {[}soil, lon, lat, time{]}  & gCm\textsuperscript{-3} \\
litr1n\_vr & 代谢凋落物氮 & {[}soil, lon, lat, time{]}  & gNm\textsuperscript{-3} \\
litr2n\_vr & 纤维素凋落物氮 & {[}soil, lon, lat, time{]}  & gNm\textsuperscript{-3} \\
litr3n\_vr & 木质素凋落物氮 & {[}soil, lon, lat, time{]}  & gNm\textsuperscript{-3} \\
soil1n\_vr & 快速周转土壤氮 & {[}soil, lon, lat, time{]}  & gNm\textsuperscript{-3} \\
soil2n\_vr & 慢速周转土壤氮 & {[}soil, lon, lat, time{]}  & gNm\textsuperscript{-3} \\
soil3n\_vr & 惰性土壤氮 & {[}soil, lon, lat, time{]}  & gNm\textsuperscript{-3} \\
cwdn\_vr & 粗木质残体氮 & {[}soil, lon, lat, time{]}  & gNm\textsuperscript{-3} \\
sminn\_vr & 无机氮 & {[}soil, lon, lat, time{]}  & gNm\textsuperscript{-3} \\
\end{longtable}

\subsection{作物模式输出变量}\label{sec_hist_vars_crop}
\begin{longtable}[htbp]{lp{0.3\textwidth}ll}
\caption{作物模式输出变量} \\

\toprule
\textbf{变量名} & \textbf{说明} & \textbf{维数} & \textbf{单位} \\\midrule
\endfirsthead
\multicolumn{4}{r}
{{\bfseries \tablename\ \thetable{} -- \kaishu 续表}} \\
\toprule
\textbf{变量名} & \textbf{说明} & \textbf{维数} & \textbf{单位} \\
\endhead
\bottomrule
\endfoot
\bottomrule
\endlastfoot

cphase & 作物物候阶段 & {[}lon, lat, time{]}  & - \\
gddpmaturity & 作物成熟积温 & {[}lon, lat, time{]} & ddays \\
gddplant & 种植日期后的积温 & {[}lon, lat, time{]} & ddays \\
vf & 作物春化响应因子 & {[}lon, lat, time{]} & - \\
hui & 作物热单位因子  & {[}lon, lat, time{]} & - \\
cropprod1c & 作物产物碳 & {[}lon, lat, time{]} & gCm\textsuperscript{-2} \\
cropprod1c\_loss & 作物产物碳消耗 & {[}lon, lat, time{]} & gCm\textsuperscript{-2}s\textsuperscript{-1} \\
cropseedc\_deficit & 作物种子碳亏缺 & {[}lon, lat, time{]} & gCm\textsuperscript{-2}s\textsuperscript{-1} \\
grainc\_to\_cropprodc & 作物收获产量 & {[}lon, lat, time{]} & gCm\textsuperscript{-2}s\textsuperscript{-1} \\
grainc\_to\_seed & 作物种子产量 & {[}lon, lat, time{]} & gCm\textsuperscript{-2}s\textsuperscript{-1} \\

\end{longtable}

\section{径流模型CaMa-Flood}
关于CaMa-Flood运行时所用到的设定可在run目录下的CaMa-Flood.nml中进行修改。以下对此进行详细介绍。
\subsection{运行时参数}
模式运行参数是模型的核心设置,它们决定了模型在执行过程中的基本行为和能力。主要的设定详见表~\ref{模式运行参数设定}

\begin{table}[htbp]
\caption{CaMa-Flood~模式运行参数设定}
\centering \renewcommand{\arraystretch}{1.5}
\label{模式运行参数设定}
\begin{tabular}{lcl}
\toprule
\textbf{变量名} & \textbf{选项} & \textbf{说明} \\\midrule

LADPSTP & 开关 & 是否采用自适应时间步长计算 \\
LFPLAIN & 开关 & 是否考虑洪泛过程 \\
LKINE & 开关 & 是否使用运动波方程 \\
LFLDOUT & 开关 & 是否开启漫滩流运动过程模拟 \\
LPTHOUT & 开关 & 是否激活河网分叉方案 \\
LDAMOUT & 开关 & 是否激活大坝水库模拟方案 \\
LROSPLIT & 开关 & 产流输入是否区分地上流和地下水流 \\
LGDWDLY & 开关 & 是否启用地下水存储及其延迟效应 \\
LSLPMIX & 开关 &
\makecell[l]{是否基于坡度激活混合水动力模式(运动波\\和局部惯性波混合)} \\
LMEANSL & 开关 & 是否在出海口使用平均海平面作为边界条件 \\
LSEALEV & 开关 & 是否在出海口使用实际观测海平面作为边界条件 \\
LRESTART & 开关 & 是否从重启文件中读取初始场 \\
LSTOONLY & 开关 & 是否从重启文件中读取水量信息 (数据同化使用) \\
LOUTPUT & 开关 & 是否使用标准输出 \\
LGRIDMAP & 开关 & 是否使用标准输出格式(基于经纬度) \\
LMAPEND & 开关 & 地图数据是否需要端转换 \\
LSTG\_ES & 开关 & for Vector Processor optimization \\
CDIMINFO & 字符串 & 输入产流的经纬度信息 \\
DT & 整型 & 模拟的时间分辨率(秒) \\
IFRQ\_INP & 整型 & 输入频率(DT的倍数) \\
LDAMOPT & 字符串 & 设定大坝水库模拟方案(如有) \\
LDAMTXT & 开关 & 是否单独输出水库位置的出入流(如有) \\
\bottomrule
\end{tabular}
\end{table}

\subsection{内部计算涉及的经验参数设定}
模式内部计算涉及的经验参数是模式运行过程中至关重要的调节因素,能够显著影响模式的性能和输出质量,详见表~\ref{CaMa模式内部计算涉及的经验参数设定}。高级用户可通过修改这些参数来对模式输出进行优化。

\begin{table}[!htbp]
\caption{CaMa-Flood模式内部计算涉及的经验参数设定}
\centering \renewcommand{\arraystretch}{1.5}
\label{CaMa模式内部计算涉及的经验参数设定}
\begin{tabular}{lcl}
\toprule
\textbf{变量名} & \textbf{数值} & \textbf{说明} \\\midrule
PMANRIV & 0.03D0 & 河道曼宁系数 \\
PMANFLD & 0.10D0 & 洪泛区曼宁系数 \\
PGRV & 9.8D0 & 重力加速度 \\
PDSTMTH & 10000.D0 & 出海口下游的距离 \\
PCADP & 0.7 & CFL的满足程度(0为无需满足,1为完全满足) \\
PMINSLP & 1.D-5 & 网格最小坡度 \\
\bottomrule
\end{tabular}
\end{table}

\subsection{读写参数设定}
读写参数设定主要对输入输出的数据的精度和命名进行设定,详见表~\ref{CaMa-Flood模式读写参数设定}。
\begin{table}[!htbp]
\caption{CaMa-Flood模式读写参数设定}
\centering \renewcommand{\arraystretch}{1.5}
\label{CaMa-Flood模式读写参数设定}
\begin{tabular}{lcl}
\toprule
\textbf{变量名} & \textbf{元素值} & \textbf{说明} \\\midrule
IMIS & -9999 & 整型缺失值 \\
RMIS & 1.E36 & 单精度浮点型缺失值 \\
DMIS & 1.E36 & 双精度浮点型缺失值 \\
CSUFBIN & \textquotesingle.bin\textquotesingle{} &
二进制二维地图数据文件后缀(字符串) \\
CSUFVEC & \textquotesingle.vec\textquotesingle{} &
二进制一维向量数据的文件后缀(字符串) \\
CSUFPTH & \textquotesingle.pth\textquotesingle{} &
二进制河网分叉数据的文件后缀(字符串) \\
CSUFCDF & \textquotesingle.nc\textquotesingle{} &
NetCDF文件的后缀(字符串) \\
CVARSOUT & & 输出 \\
\bottomrule
\end{tabular}
\end{table}

\subsection{地图文件信息设定}
地图文件信息设定主要针对模式所需的地图文件的路径和命名进行设定,详见表~\ref{CaMa-Flood模式地图文件信息设定}。
\begin{table}[!htbp]
\caption{CaMa-Flood模式地图文件信息设定}
\centering \renewcommand{\arraystretch}{1.5}
\label{CaMa-Flood模式地图文件信息设定}

\begin{tabular}{lcp{0.5\textwidth}}
\toprule
\textbf{变量名} & \textbf{选项/元素值} & \textbf{说明} \\\midrule
LMAPCDF & 开关 & 地图文件是否是netcdf格式 \\
CNEXTXY & 字符串 & 河网每一个网格的下游网格位置信息 \\
CGRAREA & 字符串 & 集水面积数据的路径和文件名 \\
CELEVTN & 字符串 & 堤坝高度数据的路径和文件名 \\
CNXTDST & 字符串 & 到河口的距离数据的路径和文件名 \\
CRIVLEN & 字符串 & 河道长度数据的路径和文件名 \\
CFLDHGT & 字符串 & 漫滩高程剖面数据的路径和文件名 \\
CRIVWTH & 字符串 & 河道宽度数据的路径和文件名 \\
CRIVHGT & 字符串 & 河道深度度数据的路径和文件名 \\
CRIVMAN & 字符串 & 河道曼宁系数数据的路径和文件名 \\
CPTHOUT & 字符串 & 河道分叉数据的路径和文件名 \\
CGDWDLY & 字符串 & 地下水延迟参数数据的路径和文件名 \\
CMEANSL & 字符串 & 平均海平面数据的路径和文件名 \\
CRIVCLINC & 字符串 & 河网netcdf数据的路径和文件名(如有) \\
CRIVPARNC & 字符串 & 河网参数netcdf数据的路径和文件名(如有) \\
CMEANSLNC & 字符串 & 平均海平面netcdf数据的路径和文件名(如有) \\
CMPIREG & 字符串 & {MPI}运算时的区域划分数据的路径和文件名(如有) \\
CDAMFILE & 字符串 & 大坝水库调水相关信息数据的路径和文件名(如有) \\
\bottomrule
\end{tabular}
\end{table}
%\clearpage

\subsection{汇流模式输出变量设定}
汇流模式输出变量设定主要用于指定输出的变量,详见表~\ref{CaMa-Flood汇流模式输出变量}。
%\begin{longtable}[htbp]{lp{0.3\textwidth}ll}
\begin{table}[htbp]
\centering \renewcommand{\arraystretch}{1.5}
\caption{CaMa-Flood汇流模式输出变量}
\label{CaMa-Flood汇流模式输出变量}
\begin{tabular}{lp{0.25\textwidth}ll}
\toprule
\textbf{变量名} & \textbf{说明} & \textbf{维数} & \textbf{单位} \\ \midrule 
%\endfirsthead
%\multicolumn{4}{r}
%{{\bfseries \tablename\ \thetable{} -- \kaishu 续表}} \\
%\toprule
%\textbf{变量名} & \textbf{说明} & \textbf{维数} & %\textbf{单位} \\ \midrule
%\endhead
%\bottomrule
%\endfoot
%\bottomrule
%\endlastfoot

rivout & 河流流量 & {[}lon, lat, time{]}  & \unit{m^3} \\
rivsto & 河流库容 & {[}lon, lat, time{]}  & \unit{m^3} \\
rivdph & 河水深度 & {[}lon, lat, time{]}  & \unit{m} \\
rivvel & 河流平均流速 & {[}lon, lat, time{]}  & \unit{m.s^{-1}} \\
fldout & 洪泛区流量 & {[}lon, lat, time{]}  & \unit{m^3.s^{-1}} \\
fldsto & 洪泛区库容 & {[}lon, lat, time{]}  & \unit{m^3} \\
flddph & 洪泛区水深 & {[}lon, lat, time{]}  & \unit{m} \\ %\textquotesingle{} \\
fldfrc & 洪泛区的网格占比 & {[}lon, lat, time{]}  & \textquotesingle0-1\textquotesingle{} \\
fldare & 洪泛区的面积 & {[}lon, lat, time{]}  & \unit{m^2} \\
sfcelv & 地表水的高度 & {[}lon, lat, time{]}  & \unit{m} \\
totout & 总径流量 & {[}lon, lat, time{]}  & \unit{m^3.s^{-1}} \\
outflw & 总径流量 & {[}lon, lat, time{]}  & \unit{m^3.s^{-1}} \\
totsto & 总库容量 & {[}lon, lat, time{]}  & \unit{m^3} \\
storge & 总库容量 & {[}lon, lat, time{]}  & \unit{m^3} \\
pthout & 分叉河道水通量 & {[}lon, lat, time{]}  & \unit{m^3.s^{-1}} \\
gdwsto & 地下水容量 & {[}lon, lat, time{]}  & \unit{m^3} \\
gwsto & 地下水容量 & {[}lon, lat, time{]}  & \unit{m^3} \\
gwout & 地下水流量 & {[}lon, lat, time{]}  & \unit{m^3.s^{-1}} \\
runoff & 地表水流量 & {[}lon, lat, time{]}  & \unit{m^3.s^{-1}} \\
runoffsub & 地表水流量 & {[}lon, lat, time{]}  & \unit{m^3.s^{-1}} \\
maxsto & 日最大库容量 & {[}lon, lat, time{]}  & \unit{m^3.s^{-1}} \\
maxflw & 日最大径流量 & {[}lon, lat, time{]}  & \unit{m^3.s^{-1}} \\
maxdph & 日最大河水深度 & {[}lon, lat, time{]}  & \unit{m} \\
damsto & 大坝库容 & {[}lon, lat, time{]}  & \unit{m^3} \\
daminf & 大坝流入水通量 & {[}lon, lat, time{]}  & \unit{m^3.s^{-1}} \\
damouf & 大坝流出水通量 & {[}lon, lat, time{]}  & \unit{m^3.s^{-1}} \\
\bottomrule
%\end{longtable}
\end{tabular}
\end{table}

\subsection{CaMa-Flood不同分辨率全球/区域地图的制作}
现有版本CoLM默认运行全球尺度的CaMa-Flood,空间分辨率为15min(15弧分)(\texttt{CaMaMap4CoLM\_glb\_0.25in\_0.25out})。如果要使用其他分辨率的河流网络地图文件(等同于运行其他分辨率的模式模拟)或者区域模拟,需要准备所需分辨率(或区域)的河流地形参数。目前,除了已有的15min数据,还有其他不同分辨率(01min、03min、05min和06min)的河流网络地图可供使用。这些高分辨率地图主要用于区域模拟,可以从 CaMa-Flood v4 网页 (\url{https://hydro.iis.u-tokyo.ac.jp/~yamadai/cama-flood/index.html})下载。以全球6min和中国区域1min为例,主要步骤介绍如下:

\begin{enumerate}
\item 目录创建

首先复制地图制作的脚本/代码(map/src)到创建的新目录, 进入该脚本目录,进行编译:\verb|make clean; make all|。

\item 区域河网数据提取

如果是全球模拟,这一步骤可以略过。针对区域模拟,则需要制作提前准备区域河网地图。CaMa-Flood 提供了从全球地图中提取区域地图的工具。需要进入\texttt{src\_region}文件夹,修改以下内容并执行\texttt{s01-regional\_map.sh}:
\begin{lstlisting}
WEST="69.985"         # 西边界
EAST="145.015"        # 东边界
SOUTH="9.985"         # 南边界
NORTH="60.115"        # 北边界
\end{lstlisting}
以上是中国区域 \ang{;1;} 模拟设定的案例。

\item 映射文件生成

进入\texttt{src\_param}文件夹, 修改\texttt{s02-generate\_inpmat.sh}:

全球的话,根据输入的runoff信息(0.1\textdegree)修改一下内容生成weight映射文件:
\begin{lstlisting}
DIMINFO="diminfo_test-06min_nc.txt" #指定维度信息文件名
INPMAT="inpmat_test-06min_nc.bin"   #指定生成的插值文件名
GRSIZEIN=0.1         #指定输入的runoff的网格分辨率,0.1代表着colm生成的runoff数据分辨率为0.1度,如果生成的runoff数据分辨率为0.25度,则需设置为0.25。
WESTIN=-180.0        #指定输入的runoff的西边界。
EASTIN=180.0         #指定输入的runoff的东边界。
NORTHIN=90.0         #指定输入的runoff的北边界。
SOUTHIN=-90.0        #指定输入的runoff的南边界。
OLAT="NtoS"          #指定输入的runoff的记录顺序,NtoS指从北到南,StoN指从南到北。
TAG="15sec"          #指定生成weight文件所使用的高分辨率数据,目前有01min,15sec和3sec三种可供选择。
\end{lstlisting}

中国区域的话,根据输入的runoff信息(0.03度)修改一下内容生成weight映射文件:
\begin{lstlisting}
DIMINFO="diminfo_China.txt"  #指定维度信息文件名
INPMAT="inpmat_China.bin"    #指定生成的插值文件名
GRSIZEIN=0.03   #指定输入的runoff的网格分辨率,0.03代表着CoLM生成的runoff数据分辨率为0.03度,如果生成的runoff数据分辨率为0.25度,则需设置为0.25。
WESTIN=69.985   #指定输入的runoff的西边界,注意不是网格中心点。
EASTIN=145.015  #指定输入的runoff的东边界,注意不是网格中心点。
NORTHIN=60.115  #指定输入的runoff的北边界,注意不是网格中心点。
SOUTHIN=9.985   #指定输入的runoff的南边界,注意不是网格中心点。
OLAT="StoN"     #指定输入的runoff的记录顺序,NtoS指从北到南,StoN指从南到北。
TAG="15sec"     #指定生成weight文件所使用的高分辨率数据,目前有01min,15sec和3sec三种可供选择。
\end{lstlisting}

修改完以上信息,就可以执行
\begin{lstlisting}
sh s02-generate_inpmat.sh
\end{lstlisting}
获取相关的映射文件。

\item 准备河道横截面参数生成所需的气候态产流数据

目前大部分地形参数则是直接从高分辨率流向图和数字高程模型(DEM)中推导出来的,而河道横截面参数(河道长度和河道深度等)是基于河道径流气候态的经验函数估算的。计算河道径流的气候态流量,需要事先准备好气候态日平均产流量。生成长时间气候态日平均产流量可以通过CaMa-Flood代码中的修改并执行\texttt{CaMa/etc/runoff\_preset}文件夹下\texttt{s01-daily\_climatology.sh}来完成,也可以用CDO软件直接生成。以下以使用CDO直接生成相关数据作为案例进行详细介绍:
\begin{enumerate}
%
\item 使用ERA5-Land的runoff制作匹配分辨率的气候态runoff数据:

如果runoff的时间分辨率高于日尺度,首先使用 \verb|cdo daymean| 将分辨率统一为日尺度,然后使用 \verb|cdo mergetime| 合并文件,将闰年日(2月29号)剔除 (\verb|cdo delfeb29|),生成气候态runoff文件(\verb|cdo yeardaymean|),如\texttt{Runoff\_climatology\_06min.nc}。气候态runoff的来源可以是ERA5-Land再分析资料,也可以是自己模式生成的长时间序列runoff数据。但如果需要运行超高分辨率的CaMa-Flood,如01min(1弧分),建议使用自行生成的高分辨率数据。


\item 制作与模式空间分辨率相匹配的气候态runoff数据:

首先展示的是全球06min网格的案例。根据需要,生成命名为\texttt{mygrid}的网格信息文件(文件命名可任意),该文件包含以下内容:
\begin{lstlisting}
gridtype = lonlat  #指定网格类型。
xsize    = 3600    #指定经向网格数。
ysize    = 1800    #指定纬向网格数。
xfirst   = -179.95 #指定经向起始网格的中心点。
xinc     = 0.1     #指定网格变化的分辨率,正值为递增,负值为递减。
yfirst   = -89.95  #指定经向起始网格的中心点。
yinc     = 0.1     #指定网格变化的分辨率,正值为递增,负值为递减。
\end{lstlisting}
然后根据\texttt{mygrid}信息,对runoff进行插值,建议使用\texttt{remapcon}(质量守恒)插值方法:
\begin{lstlisting}
cdo remapcon,mygrid Runoff_climatology_06min.nc Runoff_06min.nc
\end{lstlisting}

如果需要进行区域或者其他分辨率的模拟,则需要修改\texttt{mygrid}的相关信息。以中国区域01min(1弧分)网格的模拟为例对此进行详细介绍。首先需要修改\texttt{mygrid}的网格信息文件,该文件包含以下内容:
\begin{lstlisting}
gridtype = lonlat    #指定网格类型。
xsize    = 2501      #指定经向网格数。
ysize    = 1671      #指定纬向网格数。
xfirst   = 70.000000 #指定经向起始网格的中心点。
xinc     = 0.030000  #指定网格变化的分辨率,正值为递增,负值为递减。
yfirst   = 10.000000 #指定经向起始网格的中心点。
yinc     = 0.030000  #指定网格变化的分辨率,正值为递增,负值为递减。
\end{lstlisting}

然后根据\texttt{mygrid}信息,对runoff进行插值,同样建议使用\texttt{remapcon}(质量守恒)插值方法。如:
\begin{lstlisting}
cdo remapcon,mygrid Runoff_climatology_06min.nc Runoff_06min_regional.nc
\end{lstlisting}
\end{enumerate}

\item 河道横截面参数数据生成

修改\texttt{s01-channel\_params.sh}:
\begin{lstlisting}
DIMINFO='diminfo_China.txt'
TYPE='cdf'
CROFCDF='Runoff_06min.nc'
CROFVAR='ro'         # netCDF runoff variable name
\end{lstlisting}

当完成这些设定之后,请执行 
\begin{lstlisting}
sh s01-channel_params.sh
\end{lstlisting}
具体而言,该工具首先通过 \texttt{\$(CaMa-Flood)/map/src\_param/} 目录中的 \texttt{calc\allowbreak\_outclm\allowbreak.F90} 程序计算河道径流的气候学流量(the climatology of daily river discharge),并将数据写入输出文件 \texttt{outclm.bin}。文件中的记录1是年平均流量(\unit{m^3.s^{-1}}),即$Q_{ave}$。 其次,工具通过 \texttt{\$(CaMa-Flood)/map/} 目录中的程序 \texttt{calc\_rivwth.F90} 计算河道横截面参数(河道宽度(m),$W$和河道深度(m),$B$)。生成河道宽度 \texttt{rivwth.bin} 和河道深度 \texttt{rivhgt.bin}。这两个参数是通过以下经验公式推导出来的:
\begin{equation}
W=\max(W_{min},c_w\ast{Q_{ave}}^{p_w}+W_0)
\end{equation}
\begin{equation}
B=\max(B_{min},c_B\ast{Q_{ave}}^{p_B}+B_0)
\end{equation}
其中 $W$ 是河道宽度(m),$B$ 是河道深度(m),$Q_{ave}$ 是年平均流量 (\unit{m^3.s^{-1}})。默认参数 $W_{min}$、$c_w$、$p_w$、$W_0$、$B_{min}$、$c_B$、$p_B$、$B_0$ 已在 \texttt{s01-channel\_params.sh} 中预定义。然而,这些参数的不确定性存在较高的不确定性,因此在设置新的模拟时,建议进行广泛的校准工作(更改\texttt{s01-channel\_params.sh}的参数设定)。地图原始数据还包括了基于卫星的河道宽度数据(\texttt{“width.bin”}),但由于卫星识别小于50 m宽度的河道存在较大的误差,因此河道宽度可以结合经验参数法和卫星数据进行估算(窄河道采用经验系数法,宽河道采用卫星观测数据)。二者可以通过代码 \texttt{set\_gwdlr.F90}进行合并。CaMa-Flood模拟中推荐并默认使用这一设置。
 

\end{enumerate}


\section{高级配置}
\subsection{经纬度网格的定义}

\section{辅助工具包}
CoLM2024的辅助工具包包括案例创建、案例复制、案例代码比较、创建配置文件、案例代码更新和拷贝代码6个命令组成,这6个命令互相调用,集成不同的功能脚本。create\_newcase和create\_clone是两个案例创建功能脚本,create\_namelist,create\_script、create\_header和copy\_code是四个基础脚本,用于辅助案例创建的。update\_casecode用于版本控制后的案例代码更新。diff\_casecode用于比较两个案例之间,案例和主代码之间的差异。辅助工具包主要是基于案例的概念进行打造的,每个案例拥有独立案例目录,案例目录下包括独立的代码和配置,也就是说对于案例代码的修改,将不会影响其他案例和主代码,这大大地方便模式开发者对于版本的灵活控制。

\subsection{案例目录的结构}
\subsubsection{案例目录的结构}
执行辅助工具包将自动生成案例目录,案例目录由若干子目录和文件组成,包括代码目录(bld),运行脚本(mksrf.submit, init.submit和case.submit),模拟配置文件(namelist文件),历史文件目录(history$\slash$),重启文件目录(restart$\slash$)和地表数据目录(landdata$\slash$)。其中,代码目录包括源代码目录(bld$\slash$main$\slash$, bld$\slash$mkinidata$\slash$, bld$\slash$mksrfdata$\slash$和bld$\slash$share$\slash$,编译配置文件(bld$\slash$include$\slash$Makeoption)和宏配置文件(bld$\slash$include$\slash$define.h)。历史文件(见章节~\ref{history})、重启文件(见章节~\ref{restart})、地表数据(见章节~\ref{landdata})、namelist文件(见章节~\ref{nml})和宏配置文件(见章节~\ref{define.hux6587ux4ef6})的功能和可选项已在前面章节中详细介绍。代码目录和运行脚本是辅助运行工具包中的新内容。运行脚本用于队列系统的任务提交(如SLURM、LSF和QBS等队列系统),代码目录独立于主目录的代码,可以用于案例的代码修改和初期调试,宏配置文件也在该目录下。修改该目录下的代码或宏文件之后,可以直接编译并单独应用于该案例。

\subsection{案例的编译和运行}
\subsubsection{案例的编译}
对于案例代码的编译需要进入bld目录,并运行make命令。make前要确认编译配置文件(bld$\slash$include$\slash$Makeoption)的设置是否正确,包括netcfd库,Fortran编译器的选项和路径等(见章节~\ref{comprun})。

\subsubsection{案例目录的运行}
对于典型案例我们提供了适用于队列系统任务提交的三个脚本mksrf.submit, init.submit和case.submit,他们分别对应章节~\ref{runcolm}中提到的三个colm运行步骤:地表数据制作、初始场数据制作和主程序运行。运行前需要检查队列系统所需要核数的设置是否和运行mpirun中要求的核数一致,其他内存和队列等配置是否符合服务器的要求。

\subsection{辅助工具包的配置}

辅助工具包在使用前需要进行一次配置,指定数据代码的路径、三个脚本运行需要的核数等机器信息和队列系统需要脚本的配置格式。具体配置的简略说明见下表:

\begin{table}[!htbp]
\caption{辅助工具包预配置变量一览} \label{table_toolbox}
\centering \renewcommand{\arraystretch}{1.2}
\begin{tabular}{lcp{0.35\textwidth}}
\toprule
\textbf{1.机器配置文件:} && \\ 
run\slash machine.conf && \\
\textbf{变量名} & \textbf{示例} & \textbf{说明} \\
NProcesses\_mksrf & 672 & mksrf.submit脚本需要的核数 \\
NNodes\_mksrf & 14 & mksrf.submit需要的节点数目 \\
NTasksPerNode\_mksrf & 48 & mksrf.submit对每个节点分配的核数 \\
Memory\_mksrf & 150G & mksrf.submit运行所申请内存大小 \\
Walltime\_mksrf & 24:00:00 & mksrf.submit运行所申请的时间 \\
Queue\_mksrf & normal & mksrf.submit需要使用的队列 \\
NProcesses\_mkini & 48 &init.submit脚本需要的核数 \\
NNodes\_mkini & 14 & init.submit需要的节点数目 \\
NTasksPerNode\_mkini & 48 & init.submit对每个节点分配的核数 \\
Memory\_mkini & 150G & init.submit运行所申请内存大小 \\
Walltime\_mkini & 24:00:00 & init.submit运行所申请的时间 \\
Queue\_mksrf & normal & init.submit需要使用的队列 \\
NProcesses\_case & 48 & case.submit脚本需要的核数 \\
NNodes\_case & 14 & case.submit需要的节点数目 \\
NTasksPerNode\_case & 48 & case.submit对每个节点分配的核数 \\
Memory\_case & 150G & case.submit运行所申请内存大小 \\
Walltime\_case & 24:00:00 & case.submit运行所申请的时间 \\
Queue\_mksrf & normal & case.submit需要使用的队列 \\
ROOT & ~/CoLM202X/ & 代码个目录路径 \\
RAWDATA & ~/CLMrawdata/ & 模型所需的静态原始数据 \\
RUNTIME & ~/CoLMruntime/ & 模型所需的动态地表数据 \\
MAKEOPTION & Makeoptions & 编译配置文件名 (模板在include目录下) \\
FORCINGPATH & ~/atm/GSWP3/ &气象驱动数据地址 \\
\midrule
\textbf{2.启动脚本配置模板} && \\
run\slash batch.conf && \\
\textbf{变量名} & \textbf{数据类型} & \textbf{说明} \\DEF\_file\_mesh & 字符串 & 划分\textbf{经纬度网格单元}或\textbf{非结构网格单元}的数据文件 \\
DEF\_CatchmentMesh\_data & 字符串 & 划分\textbf{流域网格单元}的数据文件或路径 \\
DEF\_GRIDBASED\_lon\_res & 浮点型 & 定义了\textbf{经纬度网格单元}的经向分辨率,当DEF\_file\_mesh指向的文件不存在时有效 \\
DEF\_GRIDBASED\_lat\_res & 浮点型 & 定义了\textbf{经纬度网格单元}的纬向分辨率,当DEF\_file\_mesh指向的文件不存在时有效 \\
DEF\_file\_mesh\_filter & 字符串 & 使用此文件,可从模拟区域中屏蔽掉部分子区域或格点 \\

\bottomrule
\end{tabular} 
\end{table}



\subsection{案例创建命令--create\_newcase}
create\_newcase 将依照用户指定路径创建新的案例,允许用户按照典型案例配置或自定义案例进行配置的方法创建新的案例。其中包括创建namelist文件,创建运行脚本,创建头文件和代码拷贝,它们分别依赖create\_namelist、create\_script、create\_header和copy\_code四个基础脚本来实现。

创建新案例的方法有两种,典型配置创建案例和自定义配置创建案例。两种方法配置案例成功后,依然可以对案例namelist文件和宏定义define.h文件进行修改,手动修改或添加案例配置。

\subsubsection{典型配置创建案例}

典型配置创建案例语法较简单,适合初级用户快速展开模型运行。

语法:

\begin{lstlisting}
./create_newcase -n {案例运行路径} -c {典型案例配置选项}
\end{lstlisting}

案例运行路径为案例的绝对路径或相对路径,要求路径末尾没有"/"符号。典型案例配置选项目前包含12种(见表~\ref{tab:cases_config}),它分别包含了分辨率设置、区域设置、次网格类型设置、土壤模型设置、生物地球化学开关方面的设置。


例:
\begin{lstlisting}
./create_newcase -n /share/home/dq010/CoLM202X/cases/50km_PFT_VG -c Global_Grid_50km_PFT_VG
\end{lstlisting}

\begin{table}[!htbp]
\renewcommand{\arraystretch}{1.5}
\centering 
\caption{典型案例配置一览}\label{tab:cases_config}
\begin{tabular}{
cccccc} \toprule
\textbf{典型配置选项名} & \textbf{区域} & \textbf{分辨率} & \textbf{次网格} & \textbf{土壤模型} & \textbf{BGC}\\ \midrule
Global\_Grid\_50km\_PFT\_VG & 全球 & 720$\times$360 & PFT & VG & 关\\
Global\_Grid\_50km\_USGS\_VG & 全球 & 720$\times$360 & USGS & VG & 关\\
Global\_Grid\_50km\_PC\_VG & 全球 & 720$\times$360 & PC & VG & 关\\
Global\_Grid\_2x2\_PFT\_VG\_BGC & 全球 & 720$\times$360 & PC & VG & 关\\
\bottomrule
\end{tabular}
\end{table}

\subsection{案例复制命令--create\_clone}

\subsection{案例代码比较--diff\_casecode}

\subsection{案例代码比较--create\_namelist}

\subsection{案例代码比较--create\_script}

\subsection{案例代码更新--update\_casecode}

\subsection{案例代码复制--copy\_code}

\end{document}
